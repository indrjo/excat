
\begin{abstract}
  Queste note iniziano col parlare di categorie preadditive e additive
  per poi passare brevemente alle categorie abeliane. Questo tipo di
  categorie assieme a certe loro proprietà aiutano a capire il perché di
  certe scelte nel definire le categorie esatte. Segue un'introduzione
  elementare sulle categorie esatte.

  In queste note si assume la {\sc Teoria delle Categorie} che si può
  imparare da~\cite{maclane:categories}, \cite{leinster:categories}
  oppure~\cite{riehl:categories}. In lingua italiana, esistono le
  lezioni~\cite{itaca:categorie} realizzate dal {\em Progetto ItaCa}.

  Le notazioni sono in larga parte prese da queste fonti. In
  particolare, se \(\cat C\) è una categoria, allora \(\abs{\cat C}\)
  indica la sua classe degli oggetti e, per \(A, B \in \abs{\cat C}\), con
  \(\cat C(A, B)\) oppure \(\hom(A, B)\) indichiamo la classe dei
  morfismi \(A \to B\). L'operazione di composizione è indicata con
  \(\circ\) oppure il simbolo è del tutto omesso scrivendo \(gf\) al posto
  di \(g \circ f\).
\end{abstract}
  
%%% Local Variables:
%%% mode: LaTeX
%%% TeX-master: "main"
%%% TeX-engine: luatex
%%% ispell-local-dictionary: "italian"
%%% End:
