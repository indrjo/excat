
\chapter{Preliminari}

In queste note si assume almeno la {\sc Teoria delle Categorie} che si
può imparare da~\cite{leinster:categories}. Un altro validissimo testo è
anche~\cite{riehl:categories}. Un vero e proprio classico che contiene
anche nozioni sulle categorie preadditive è~\cite{maclane:categories}



\section{Kernel e cokernel}

Noi lavoreremo con categorie in cui sono presenti oggetti che sono sia
terminali che iniziali. Oggetti di questo tipo sono detti {\em oggetti
  zero}. Non si tratta di categorie come \(\Set\), \(\Top\) e
\(\Ring\), ma è il caso di \(\Grp\) e \(\Modu_R\), con \(R\) anello.

\begin{definition}
  Sia \(\cat C\) una categoria con {\em oggetto zero} che indichiamo
  con \(0\). Per \(a, b \in \obj{\cat C}\), il {\em morfismo nullo} o
  {\em zero} da \(a\) a \(b\) è la composizione dei morfismi
  \[
    \begin{tikzcd}[column sep=small]
      a \ar["{\exists!}", r] & 0 \ar["{\exists!}", r] & b
    \end{tikzcd}
  \]
  Scriveremo questo morfismo come \(0_a^b\) oppure, se dal contesto è
  chiaro, semplicemente \(0\).
\end{definition}

\begin{example}
  In casi come \(\Grp\) e \(\Modu_R\), il morfismo nullo è quello che
  manda tutti gli elementi del dominio nell'identità. Anche \(\Ring\)
  possiede omomorfismi come questi; tuttavia in \(\Ring\) l'oggetto
  iniziale è \(\mathbb{Z}\) che non è terminale mentre i morfismi zero sono
  definiti attraverso oggetti zero.
\end{example}

Comporre con un morfismo nullo dà un morfismo nullo. Più precisamente:

\begin{proposition}\label{lemma:CompCon0}
  Sia \(\cat C\) una categoria con {\em oggetto zero} \(0\), e
  \(a, b, c \in \obj{\cat C}\). Allora per ogni \(f : b \to c\) di
  \(\cat C\) si ha
  \[
    f \circ 0_a^b = 0_a^c
  \]
  e per ogni \(g : c \to a\) di \(\cat C\) si ha
  \[
    0_a^b \circ g = 0_c^b .
  \]
\end{proposition}

\begin{proof}
  Proviamo solo una delle due uguaglianze perché l'altra è
  simile. Consideriamo i morfismi
  \[
    \begin{tikzcd}[column sep=small]
      a \ar["{\exists!}", r] & 0 \ar["{\exists!}", r] & b \ar["f", r] & c
    \end{tikzcd}
  \]
  Poiché \(0\) è iniziale, la composizione delle ultime due frecce è
  l'unica che può esserci. Quindi \(f \circ 0_a^b\) è la composizione
  della freccia \(a \to 0\) con la freccia \(0 \to c\), ed abbiamo
  concluso.
\end{proof}

\begin{definition}[Kernel e cokernel]
  In una categoria \(\cat C\) con oggetto zero \(0\), il {\em kernel}
  di \(f : a \to b\) è uno qualsiasi degli equalizzatori di
  \[
    \begin{tikzcd}
      a \ar["f", shift left, r] \ar["{0_a^b}", shift right, swap, r] &
      b
    \end{tikzcd}
  \]
  Dualmente, il {\em cokernel} di \(f : a \to b\) è uno qualsiasi dei
  coequalizzatori della stessa coppia di morfismi.
\end{definition}

\begin{remark}
  Ricordiamo che in generale gli equalizzatori sono monomorfismi e i
  coequalizzatori sono epimorfismi. Questa informazione può dare delle
  indicazioni su come sono fatti i kernel e i cokernel in categorie in
  cui monomorfismo ed epimorfismo significano rispettivamente
  iniettivo e suriettivo.
\end{remark}

\begin{example}[Kernel di omomorfismi di moduli]\label{example:KernelInModR}
  In Algebra, si parla di {\em kernel} di morfismi in vari ambiti. Per
  esempio, se \(M\) e \(N\) sono due moduli su un fissato anello \(R\)
  e \(f : M \to N\) è un omomorfismo, allora
  \[
    \ker f := \set{x \in M \mid f(x) = 0_N}
  \]
  è un sottomodulo di \(M\). Per entrare nel linguaggio della {\sc
    Teoria delle Categorie},
  \begin{quotation}
    l'omomorfismo inclusione \(i : \ker f \hookrightarrow M\) è un equalizzatore
    della coppia di omomorfismi
    \[
      \begin{tikzcd}
        M \ar["f", shift left, r] \ar["{0_M^N}", shift right, swap, r]
        & N
      \end{tikzcd}
    \]
  \end{quotation}
  % Verifichiamo l'affermazione. Anzitutto, è facile vedere che
  % \(f \circ i = 0_M^N \circ i\), poiché per la
  % Proposizione~\ref{lemma:CompCon0} si ha che
  % \(0_M^N \circ i = 0_{\ker f}^N\). Consideriamo ora invece un qualsiasi
  % diagramma commutativo
  % \[
  %   \begin{tikzcd}
  %     E \ar["j", r] & M \ar["f", shift left, r] \ar["{0_M^N}", shift
  %     right, swap, r] & N
  %   \end{tikzcd}
  % \]
  % Questo implica che \(f(j(x)) = 0_M^N (j(x)) = 0_N\) per ogni
  % \(x \in E\). Questo significa che le immagini di \(j\) stanno in
  % \(\ker f\), quindi è possibile la fattorizzazione \(j = i \circ k\) dove
  % \(k : E \hookrightarrow \ker f\) è una semplice inclusione. Per finire
  % \(k\) è l'unico omomorfismo \(E \to \ker f\) per cui commuta il
  % triangolo
  % \[
  %   \begin{tikzcd}
  %     \ker f \ar["i", r] & M \\
  %     E \ar[u] \ar["j", ur, swap]
  %   \end{tikzcd}
  % \]
  % a causa dell'iniettività di \(i\).
\end{example}

\begin{example}[Cokernel di omomorfismi di moduli]\label{example:CokernelInModR}
  Se \(M\) e \(N\) sono due moduli su un fissato anello \(R\) e
  \(f : M \to N\) è un omomorfismo, allora
  \[
    \coker f := \frac{N}{\im f}
  \]
  è un sottomodulo di \(M\) chiamato {\em cokernel} di \(f\). Questa
  nozione è il duale di kernel, nel senso che
  \begin{quotation}
    l'omomorfismo di proiezione canonica al quoziente
    \(\pi_N : N \to \coker f\) è un coequalizzatore della coppia di
    omomorfismi
    \[
      \begin{tikzcd}
        M \ar["f", shift left, r] \ar["{0_M^N}", shift right, swap, r]
        & N
      \end{tikzcd}
    \]
  \end{quotation}
  % Per dimostrare questo fatto si può usare un teorema di omomorfismo,
  % oppure un argomento categoriale come questo.
  % \begin{enumerate}
  % \item Considerare il funtore dimenticante \(U : \Modu_R \to \Set\) che
  %   è fedele.
  % \item In \(\Set\) un coequalizzatore di
  %   \[
  %     \begin{tikzcd}
  %       X \ar["f", r, shift left] \ar["g", r, shift right, swap] & Y
  %     \end{tikzcd}
  %   \]
  %   è la proiezione canonica al quoziente \(p : Y \to Y{/}\sim\) dove
  %   \(\sim\) è la relazione di equivalenza generata da
  %   \(f(a) \sim f(b)\) al variare di \(a, b \in X\).
  % \item Se \(M\) è un modulo su \(R\) e \(H\) è un suo sottomodulo, il
  %   modulo quoziente \(\frac{M}{H}\) è ottenuto quozientando \(M\) con
  %   la relazione di equivalenza \(\sim\) generata da \(x+h \sim x\) per
  %   \(x \in M, h \in H\).
  % \end{enumerate}
\end{example}

\begin{remark}
  In Algebra, kernel e cokernel sono degli oggetti, mentre
  tecnicamente come sono definiti qui sono dei morfismi.
\end{remark}



\section{Categorie preadditive}

\begin{definition}
  Una {\em categoria preadditiva} è una categoria \(\cat C\) in cui:
  \begin{enumerate}[leftmargin=*]
  \item Per ogni \(a, b \in \obj{\cat C}\) la classe \(\cat C(a, b)\) è
    dotata di un'operazione interna
    \[
      +_{a,b} : \cat C (a, b) \times \cat C (a, b) \to \cat C (a, b)
    \]
    e ha un elemento \(0_a^b : a \to b\) che lo rendono un gruppo
    abeliano.
  \item Per ogni \(a, b, c \in \obj{\cat C}\) e \(f : a \to b\) di
    \(\cat C\), le funzioni
    \begin{align*}
      & f_\ast := \cat C(c, f) : \cat C(c, a) \to \cat C(c, b) \\
      & f^\ast := \cat C(f, c) : \cat C(b, c) \to \cat C(a, c)
    \end{align*}
    sono omomorfismi di gruppi abeliani.
  \end{enumerate}
\end{definition}

\begin{recall}
  Ricordiamo che \(f_\ast(g) := f \circ g\) mentre \(f^\ast(h) := h \circ f\).
\end{recall}

Spesso scriveremo semplicemente \(+\) senza pedici, perché in genere è
chiaro di quali frecce stiamo sommando.

Inoltre, come nei primi teoremi di Algebra, il morfismo zero è l'unico
elemento neutro e per ogni \(f : a \to b\) è unico l'opposto. In
coerenza con la notazione additiva indichiamo con \(- f\) l'opposto di
\(f : a \to b\).

Un'altra osservazione da fare è questa. Il simbolo \(0_a^b\) in una
categoria \(\cat C\) con oggetto zero indica il morfismo zero
\(a \to b\). Se \(\cat C\) è preaddittiva, non sono notazioni in
conflitto?  No.

\begin{proposition}
  In una categoria preadditiva \(\cat C\) con oggetto zero \(0\), il
  morfismo nullo è elemento neutro.
\end{proposition}

\begin{proof}
  Poiché \(0\) è iniziale, \(\cat C(0, b)\) è banale e in particolare
  \(0_0^b\) è elemento neutro. L'omomorfismo
  \[
    \left(0_a^0\right)^\ast : \cat C(0, b) \to \cat C(a, b)
  \]
  manda l'elemento neutro del dominio in quello del codominio, che è
  \[
    \left(0_a^0\right)^\ast\left(0_0^b\right) = 0_0^b \circ 0_a^0 = 0_a^b
    . \qedhere
  \]
\end{proof}

\begin{remark}
  A tal proposito è utile osservare che se \(\cat C\) è una categoria
  preadditiva, anche il suo duale \(\opcat C\) lo è. La conseguenza più
  pratica per noi è che dimezza le dimostrazioni: una volta dimostrato
  un enunciato, quello duale è automatico.
\end{remark}

\begin{proposition}
  In un categoria preadditiva \(\cat C\) gli oggetti terminali sono
  iniziali e viceversa.
\end{proposition}

Quindi in una categoria preadditiva gli oggetti terminali e iniziali
sono oggetti zero.

\begin{proof}
  Sia \(t\) un oggetto terminale di \(\cat C\). Se riusciamo a
  mostrare che \(\cat C(t, a)\) è un gruppo banale per ogni oggetto
  \(a\) di \(\cat C\), allora possiamo concludere.\newline Prendiamo un
  \(f : t \to a\) qualsiasi in \(\cat C\) e guardiamo l'omomorfismo di
  gruppi
  \[
    f_\ast = \cat C(t, f) : \cat C(t, t) \to \cat C(t, a) .
  \]
  Il dominio è un gruppo banale perché \(t\) è terminale; in
  particolare, \(\id_t = 0_t^t\). Inoltre, trattandosi di omomorfismo,
  \[
    \underbrace{f_\ast \left(\id_t\right)}_{= f} = f\left(0_t^t\right) =
    0_t^a . \qedhere
  \]
\end{proof}

Vediamo i prodotti e i coprodotti finiti ora.

\begin{proposition}\label{proposition:BinProdsAreBinCoprods}
  Sia \(\cat C\) una categoria preadditiva con oggetto zero \(0\) e
  \[
    \begin{tikzcd}[column sep=small]
      a & a \times b \ar["{p_a}", l, swap] \ar["{p_b}", r] & b
    \end{tikzcd}
  \]
  un prodotto in \(\cat C\). Grazie alla proprietà universale di
  prodotto, introduciamo le frecce \(i_a : a \to a \times b\) e
  \(i_b : b \to a \times b\) come le uniche frecce di \(\cat C\) che fanno
  commutare i diagrammi
  \[
    \begin{tikzcd}
      & a \ar["{\id_a}", dl, swap] \ar[d] \ar["{0_a^b}", dr] \\
      a & a \times b \ar["{p_a}", l] \ar["{p_b}", r, swap] & b
    \end{tikzcd}
    %
    \quad
    %
    \begin{tikzcd}
      & b \ar["{0_b^a}", dl, swap] \ar[d] \ar["{\id_b}", dr] \\
      a & a \times b \ar["{p_a}", l] \ar["{p_b}", r, swap] & b
    \end{tikzcd}
  \]
  Allora
  \[
    \begin{tikzcd}[column sep=small]
      a \ar["{i_a}", r] & a \times b & b \ar["{i_b}", l, swap]
    \end{tikzcd}
  \]
  è coprodotto in \(\cat C\). Dualmente, se
  \[
    \begin{tikzcd}[column sep=small]
      a \ar["{i_a}", r] & a + b & b \ar["{i_b}", l, swap]
    \end{tikzcd}
  \]
  é un coprodotto e se \(p_a : a+b \to a\) e \(p_b : a+b \to b\) sono
  introdotti attraverso la proprietà universale di coprodotto come le
  uniche frecce che fanno commutare i diagrammi
  \[
    \begin{tikzcd}
      & a \\
      a \ar["{i_a}", r, swap] \ar["{\id_a}", ur] & a + b \ar[u] & b
      \ar["{i_b}", l] \ar["{0_b^a}", ul, swap]
    \end{tikzcd}
    %
    \quad
    %
    \begin{tikzcd}
      & a \\
      a \ar["{i_a}", r, swap] \ar["{\id_a}", ur] & a + b \ar[u] & b
      \ar["{i_b}", l] \ar["{0_b^a}", ul, swap]
    \end{tikzcd}
  \]
  allora
  \[
    \begin{tikzcd}[column sep=small]
      a & a + b \ar["{p_a}", l, swap] \ar["{p_b}", r] & b
    \end{tikzcd}
  \]
  è un prodotto in \(\cat C\).
\end{proposition}

In breve: in categorie preadditive si ha \(a \times b \cong a+b\).

\begin{proof}
  Possiamo limitarci a dimostrare solo il primo fatto. Consideriamo un
  qualsiasi oggetto con morfismi
  \[
    \begin{tikzcd}[column sep=small]
      a \ar["f", r] & c & b \ar["g", l, swap]
    \end{tikzcd}
  \]
  e cerchiamo un modo di costruire un morfismo \(a \times b \to c\). Le
  frecce \(f \circ p_a\) e \(g \circ p_b\) sono di questo tipo, ma non vanno
  bene per i nostri scopi. Invece la somma sì, perché il diagramma
  \[
    \begin{tikzcd}[row sep=large]
      & c \\
      a \ar["{i_a}", r, swap] \ar["f", ur, bend left] & a + b \ar["{f \circ p_a + g \circ
        p_b}" description, u] & b
      \ar["{i_b}", l] \ar["g", ul, bend right, swap]
    \end{tikzcd}
  \]
  commuta.
\end{proof}

\begin{definition}\label{definition:Biprodotto}
  Un {\em biprodotto} in una categoria preadditiva consta di oggetti e frecce
  \[
    \begin{tikzcd}
      a \ar["{i_a}", r, shift right, swap] & c \ar["{p_a}", l, shift
      right, swap] \ar["{p_b}", r, shift left] & b \ar["{i_b}", l,
      shift left]
    \end{tikzcd}
  \]
  tali che
  \begin{align}
    & p_a \circ i_a = \id_a \label{eqn:Biprod1} \\
    & p_b \circ i_b = \id_b \label{eqn:Biprod2}\\
    & i_a \circ p_a + i_b \circ p_b = \id_c \label{eqn:Biprod3}
  \end{align}
\end{definition}

\begin{remark}
  Dalle identità~\ref{eqn:Biprod1}, \ref{eqn:Biprod2}
  e~\ref{eqn:Biprod3} discendono
  \begin{align}
    & p_a \circ i_b = 0_b^a \label{eqn:Biprod4} \\
    & p_b \circ i_a = 0_a^b \label{eqn:Biprod5}
  \end{align}
  %Combinando~\ref{eqn:Biprod1} e~\ref{eqn:Biprod3} si
  %ha
  Proviamo solo la prima visto che l'altra si fa similmente.
  \[
    p_a \underbrace{=}_{\eqref{eqn:Biprod3}} p_a \circ \left( i_a \circ p_a + i_b \circ p_b \right) \underbrace{=}_{\eqref{eqn:Biprod1}} p_a + p_a \circ i_b \circ p_b
  \]
  da cui
  \[
    p_a \circ i_b \circ p_b = 0_c^a .
  \]
  Postcomponendo con \(i_b\) e usando~\eqref{eqn:Biprod2}, concludiamo \(p_a \circ i_b = 0_b^a\).
\end{remark}

\begin{proposition}
  In una categoria preadditiva con oggetto zero \(0\), un prodotto
  \[
    \begin{tikzcd}[column sep=small]
      a & a \times b \ar["{p_a}", l, swap] \ar["{p_b}", r] & b
    \end{tikzcd}
  \]
  dà un biprodotto
  \[
    \begin{tikzcd}
      a \ar["{i_a}", r, shift right, swap] & a \times b \ar["{p_a}", l, shift
      right, swap] \ar["{p_b}", r, shift left] & b \ar["{i_b}", l,
      shift left]
    \end{tikzcd}
  \]
  in cui \(i_a\) e \(i_b\) sono introdotti come nella proposizione
  precedente. Dualmente, un coprodotto
  \[
    \begin{tikzcd}[column sep=small]
      a \ar["{i_a}", r] & a + b & b \ar["{i_b}", l, swap]
    \end{tikzcd}
  \]
  dà un biprodotto
  \[
    \begin{tikzcd}
      a \ar["{i_a}", r, shift right, swap] & a+b \ar["{p_a}", l, shift
      right, swap] \ar["{p_b}", r, shift left] & b \ar["{i_b}", l,
      shift left]
    \end{tikzcd}
  \]
  in cui \(p_a\) e \(p_b\) sono introdotti come nella proposizione
  precedente. Viceversa, se si ha un biprodotto
  \[
    \begin{tikzcd}
      a \ar["{i_a}", r, shift right, swap] & a+b \ar["{p_a}", l, shift
      right, swap] \ar["{p_b}", r, shift left] & b \ar["{i_b}", l,
      shift left]
    \end{tikzcd}
  \]
  allora la coppia di frecce \(p_a\) e \(p_b\) è prodotto e la
  coppia \(i_a\) e \(i_b\) è coprodotto.
\end{proposition}

Quindi in categorie preadditive esiste un modo piuttosto compatto di
esprimere prodotti e coprodotti (che sono la stessa cosa) senza
passare per le proprietà universali.

\begin{proof}
  Verifichiamo ad esempio che i prodotti danno biprodotti. Le
  identità~\eqref{eqn:Biprod1} e~\eqref{eqn:Biprod2} discendono
  direttamente da come sono introdotte \(i_a\) e \(i_b\). Consideriamo
  ora il morfismo
  \[
    i_a \circ p_a + i_b \circ p_b : a \times b \to a \times b .
  \]
  Componendo una volta con \(p_a\) e l'altra con \(p_b\) si ha (vedi
  anche~\eqref{eqn:Biprod4} e~\eqref{eqn:Biprod5})
  \begin{align*}
    & p_a \circ \left(i_a \circ p_a + i_b \circ p_b\right) = \underbrace{p_a \circ
      i_a}_{= \id_a} \circ p_a + \underbrace{p_a
      \circ i_b}_{= 0_b^a} \circ p_b = p_a \\
    & p_b \circ \left(i_a \circ p_a + i_b \circ p_b\right) = \underbrace{p_b \circ
      i_a}_{= 0_a^b} \circ p_a + \underbrace{p_b
      \circ i_b}_{= \id_b} \circ p_b = p_b
  \end{align*}
  Per la proprietà universale di prodotto, si può concludere che
  \[
    i_a \circ p_a + i_b \circ p_b = \id_{a \times b} .
  \]
  Per il viceversa prendendo qualsiasi coppia di morfismi
  \[
    \begin{tikzcd}[column sep=small]
      a & a \times b \ar["f", l, swap] \ar["g", r] & b
    \end{tikzcd}
  \]
  si ha
  \begin{align*}
    & p_a \circ \left( i_a \circ f + i_b \circ g \right) = f \\
    & p_b \circ \left( i_a \circ f + i_b \circ g \right) = g
  \end{align*}
  Ora se \(h : c \to a \times b\) è tale che \(p_a \circ h = f\) e \(p_b \circ h =
  g\), allora
  \[
    h = \id_{a \times b} \circ h = \left( i_a \circ p_a + i_b \circ p_b \right) \circ h =
    i_a \circ f + i_b \circ g . \qedhere 
  \]
\end{proof}

\begin{proposition}
  In una categoria preadditiva \(\cat C\) con oggetto zero \(0\), gli
  equalizzatori di
  \[
    \begin{tikzcd}
      a \ar["f", r, shift left] \ar["g", r, shift right, swap] & b
    \end{tikzcd}
  \]
  sono equalizzatori di
  \[
    \begin{tikzcd}
      a \ar["{f-g}", r, shift left] \ar["{0_a^b}", r, shift right, swap] & b
    \end{tikzcd}
  \]
  e viceversa. Analogamente vale per i coequalizzatori.
\end{proposition}

Quindi un (co)equalizzatore di una coppia di morfismi è un (co)kernel della
differenza dei due. 

\begin{proof}
  Basta osservare che
  \[
    \begin{tikzcd}
      e \ar["i", r] & a \ar["f", r, shift left] \ar["g", r, shift right, swap] & b
    \end{tikzcd}
  \]
  commuta se e solo se commuta
  \[
    \begin{tikzcd}
      e \ar["i", r] & a \ar["{f-g}", r, shift left] \ar["{0_a^b}", r, shift right, swap] & b
    \end{tikzcd}
  \qedhere\]
\end{proof}



\section{Categorie additive}



%%% Local Variables:
%%% mode: LaTeX
%%% TeX-master: "../main"
%%% TeX-engine: luatex
%%% ispell-local-dictionary: "italian"
%%% End:
