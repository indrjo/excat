
\chapter{Preliminari}

In queste note si assume almeno la {\sc Teoria delle Categorie} che si
può imparare da~\cite{leinster:categories}. Un altro validissimo testo
che insegna le basi è anche~\cite{riehl:categories}. Nozioni sulle
categorie preadditive si possono trovare in~\cite{maclane:categories}
che è considerato un vero e proprio classico.


\section{Kernel e cokernel}

Noi lavoreremo con categorie in cui sono presenti oggetti che sono sia
terminali che iniziali. Oggetti di questo tipo sono detti {\em oggetti
  zero}. Non si tratta di categorie come \(\Set\), \(\Top\) e
\(\Ring\), ma è il caso di \(\Grp\) e \(\Modu_R\), con \(R\) anello.

\begin{definition}
  Sia \(\cat C\) una categoria con {\em oggetto zero} che indichiamo
  con \(0\). Per \(a, b \in \obj{\cat C}\), il {\em morfismo nullo} o
  {\em zero} da \(a\) a \(b\) è la composizione dei morfismi
  \[
    \begin{tikzcd}[column sep=small]
      a \ar["{\exists!}", r] & 0 \ar["{\exists!}", r] & b
    \end{tikzcd}
  \]
  Scriveremo questo morfismo come \(0_a^b\) oppure, se dal contesto è
  chiaro, semplicemente \(0\).
\end{definition}

\begin{example}
  In casi come \(\Grp\) e \(\Modu_R\), il morfismo nullo è quello che
  manda tutti gli elementi del dominio nell'identità. Anche \(\Ring\)
  possiede omomorfismi come questi; tuttavia in \(\Ring\) l'oggetto
  iniziale è \(\mathbb{Z}\) che non è terminale mentre i morfismi zero sono
  definiti attraverso oggetti zero.
\end{example}

Comporre con un morfismo nullo dà un morfismo nullo. Più precisamente:

\begin{proposition}\label{lemma:CompCon0}
  Sia \(\cat C\) una categoria con {\em oggetto zero} \(0\), e
  \(a, b, c \in \obj{\cat C}\). Allora per ogni \(f : b \to c\) di
  \(\cat C\) si ha
  \[
    f \circ 0_a^b = 0_a^c
  \]
  e per ogni \(g : c \to a\) di \(\cat C\) si ha
  \[
    0_a^b \circ g = 0_c^b .
  \]
\end{proposition}

\begin{proof}
  Proviamo solo una delle due uguaglianze perché l'altra è
  simile. Consideriamo i morfismi
  \[
    \begin{tikzcd}[column sep=small]
      a \ar["{\exists!}", r] & 0 \ar["{\exists!}", r] & b \ar["f", r] & c
    \end{tikzcd}
  \]
  Poiché \(0\) è iniziale, la composizione delle ultime due frecce è
  l'unica che può esserci. Quindi \(f \circ 0_a^b\) è la composizione
  della freccia \(a \to 0\) con la freccia \(0 \to c\), ed abbiamo
  concluso.
\end{proof}

\begin{definition}[Kernel e cokernel]
  In una categoria \(\cat C\) con oggetto zero \(0\), il {\em kernel}
  di \(f : a \to b\) è uno qualsiasi degli equalizzatori di
  \[
    \begin{tikzcd}
      a \ar["f", shift left, r] \ar["{0_a^b}", shift right, swap, r] &
      b
    \end{tikzcd}
  \]
  Dualmente, il {\em cokernel} di \(f : a \to b\) è uno qualsiasi dei
  coequalizzatori della stessa coppia di morfismi.
\end{definition}

\begin{remark}
  Ricordiamo che in generale gli equalizzatori sono monomorfismi e i
  coequalizzatori sono epimorfismi. Questa informazione può dare delle
  indicazioni su come sono fatti i kernel e i cokernel in categorie in
  cui monomorfismo ed epimorfismo significano rispettivamente
  iniettivo e suriettivo.
\end{remark}

\begin{example}[Kernel di omomorfismi di moduli]
  In Algebra, si parla di {\em kernel} di morfismi in vari ambiti. Per
  esempio, se \(M\) e \(N\) sono due moduli su un fissato anello \(R\)
  e \(f : M \to N\) è un omomorfismo, allora
  \[
    \ker f := \set{x \in M \mid f(x) = 0_N}
  \]
  è un sottomodulo di \(M\). Per entrare nel linguaggio della {\sc
    Teoria delle Categorie},
  \begin{quotation}
    l'omomorfismo inclusione \(i : \ker f \hookrightarrow M\) è un equalizzatore
    della coppia di omomorfismi
    \[
      \begin{tikzcd}
        M \ar["f", shift left, r] \ar["{0_M^N}", shift right, swap, r]
        & N
      \end{tikzcd}
    \]
  \end{quotation}
  % Verifichiamo l'affermazione. Anzitutto, è facile vedere che
  % \(f \circ i = 0_M^N \circ i\), poiché per la
  % Proposizione~\ref{lemma:CompCon0} si ha che
  % \(0_M^N \circ i = 0_{\ker f}^N\). Consideriamo ora invece un qualsiasi
  % diagramma commutativo
  % \[
  %   \begin{tikzcd}
  %     E \ar["j", r] & M \ar["f", shift left, r] \ar["{0_M^N}", shift
  %     right, swap, r] & N
  %   \end{tikzcd}
  % \]
  % Questo implica che \(f(j(x)) = 0_M^N (j(x)) = 0_N\) per ogni
  % \(x \in E\). Questo significa che le immagini di \(j\) stanno in
  % \(\ker f\), quindi è possibile la fattorizzazione \(j = i \circ k\) dove
  % \(k : E \hookrightarrow \ker f\) è una semplice inclusione. Per finire
  % \(k\) è l'unico omomorfismo \(E \to \ker f\) per cui commuta il
  % triangolo
  % \[
  %   \begin{tikzcd}
  %     \ker f \ar["i", r] & M \\
  %     E \ar[u] \ar["j", ur, swap]
  %   \end{tikzcd}
  % \]
  % a causa dell'iniettività di \(i\).
\end{example}

\begin{example}[Cokernel di omomorfismi di moduli]
  Se \(M\) e \(N\) sono due moduli su un fissato anello \(R\) e
  \(f : M \to N\) è un omomorfismo, allora
  \[
    \coker f := \frac{N}{\im f}
  \]
  è un sottomodulo di \(M\) chiamato {\em cokernel} di \(f\). Questa
  nozione è il duale di kernel, nel senso che
  \begin{quotation}
    l'omomorfismo di proiezione canonica al quoziente
    \(\pi_N : N \to \coker f\) è un coequalizzatore della coppia di
    omomorfismi
    \[
      \begin{tikzcd}
        M \ar["f", shift left, r] \ar["{0_M^N}", shift right, swap, r]
        & N
      \end{tikzcd}
    \]
  \end{quotation}
  % Per dimostrare questo fatto si può usare un teorema di omomorfismo,
  % oppure un argomento categoriale come questo.
  % \begin{enumerate}
  % \item Considerare il funtore dimenticante \(U : \Modu_R \to \Set\) che
  %   è fedele.
  % \item In \(\Set\) un coequalizzatore di
  %   \[
  %     \begin{tikzcd}
  %       X \ar["f", r, shift left] \ar["g", r, shift right, swap] & Y
  %     \end{tikzcd}
  %   \]
  %   è la proiezione canonica al quoziente \(p : Y \to Y{/}\sim\) dove
  %   \(\sim\) è la relazione di equivalenza generata da
  %   \(f(a) \sim f(b)\) al variare di \(a, b \in X\).
  % \item Se \(M\) è un modulo su \(R\) e \(H\) è un suo sottomodulo, il
  %   modulo quoziente \(\frac{M}{H}\) è ottenuto quozientando \(M\) con
  %   la relazione di equivalenza \(\sim\) generata da \(x+h \sim x\) per
  %   \(x \in M, h \in H\).
  % \end{enumerate}
\end{example}

\begin{remark}
  In Algebra, kernel e cokernel sono degli oggetti, mentre
  tecnicamente come sono definiti qui sono dei morfismi.
\end{remark}



\section{Categorie preadditive}


\section{Limiti e colimiti in categorie preadditive}


\section{Categorie additive}



%%% Local Variables:
%%% mode: LaTeX
%%% TeX-master: "../main"
%%% TeX-engine: luatex
%%% ispell-local-dictionary: "italian"
%%% End:
