
\chapter{Categorie esatte}

In questo capitolo raccogliamo una serie di risultati elementari sulle
successioni esatte corte, seguendo da vicino la trattazione che si può
trovare in testi {\em elementari} come~\cite{buehler:exactcategories}.



\section{Coppie kernel-cokernel}

\begin{definition}\label{definition:KerCoker}
  Una {\em coppia kernel-cokernel} in una categoria additiva \(\cat C\)
  consiste di due morfismi consecutivi
  \[
    \begin{tikzcd}[column sep=small]
      A \ar["i", r] & B \ar["p", r] & C
    \end{tikzcd}
  \]
  tali che \(i\) è un kernel di \(p\) e \(p\) è un cokernel di \(i\).
\end{definition}

\begin{example}
  Sia in \(\Modu_R\) un omomorfismo \(f : M \to N\) e consideriamo la
  proiezione al quoziente
  \[
    \pi : M \to M/\ker f .
  \]
  Come si è visto (Esempio~\ref{example:KernelInModR}) il kernel è
  \(\ker \pi = \ker f\) con l'inclusione. Viceversa
  \[
    i : \ker f \hookrightarrow M
  \]
  ha come cokernel \(\pi\) (Esempio~\ref{example:CokernelInModR}). Quindi
  \[
    \begin{tikzcd}[column sep=small]
      \ker f \ar["i", hookrightarrow, r] & M \ar["\pi", r] & \frac{M}{\ker
        f}
    \end{tikzcd}
  \]
  è una coppia ker-coker in \(\Modu_R\). In generale, se \(H\) è un
  sottomodulo di \(M\), allora
  \[
    \begin{tikzcd}[column sep=small]
      H \ar[hookrightarrow, r] & M \ar["{\pi_H}", r] & \frac{M}{H}
    \end{tikzcd}
  \]
  è una coppia ker-coker.
\end{example}

I seguenti lemmi sono in preparazione alla successiva definizione. Sono
fatti la cui evidenza è sbalorditiva quando si tiene a mente una
categoria additiva come \(\Modu_R\).

Qual è il kernel dell'omomorfismo \(\id_M : M \to M\)? Il modulo banale. E
il cokernel di \(0 \to M\)? È \(\frac{M}{\{0\}} \cong M\). Il kernel di
\(M \to 0\)? \(M\) stesso. Il cokernel di \(\id_M : M \to M\)?
\(\frac{M}{M} \cong 0\). Ecco come si scrive in un contesto astratto come
questo.

\begin{lemma}\label{lemma:KerCokerId}
  In una categoria additiva \(\cat C \),
  \[
    \begin{tikzcd}[column sep=small]
      0 \ar["{\exists!}", r] & A \ar["{\id_A}", r] & A
    \end{tikzcd}
  \]
  è una coppia ker-coker. Dualmente,
  \[
    \begin{tikzcd}[column sep=small]
      A \ar["{\id_A}", r] & A \ar["{\exists!}", r] & 0
    \end{tikzcd}
  \]
  è una coppia ker-coker
\end{lemma}

\begin{proof}
  Il diagramma
  \[
    \begin{tikzcd}
      0 \ar["{\exists!}", r] & A \ar["{\id_A}", r, shift left] \ar["{0_A^A}",
      r, shift right, swap] & A
    \end{tikzcd}
  \]
  commuta perché \(\cat C(0,A)\) è banale, essendo \(0\) iniziale. Sia
  \(i : E \to A\) tale che
  \[
    \begin{tikzcd}
      E \ar["{i}", r] & A \ar["{\id_A}", r, shift left] \ar["{0_A^A}",
      r, shift right, swap] & A
    \end{tikzcd}
  \]
  commuta, cioè \(i = 0_E^A \). Essendo 0 terminale, c'è esattamente un
  \(E \to 0\). Il triangolo
  \[
    \begin{tikzcd}
      0 \ar["{\exists!}", r] & A \\
      E \ar["{\exists!}", u] \ar["i", ur, swap]
    \end{tikzcd}
  \]
  commuta per come sono definiti i morfismi nulli.  Verifichiamo ora che
  \(\id_A\) è cokernel di \(0 \to A\). Anzitutto
  \[
    \begin{tikzcd}
      0 \ar["{\exists!}", r, shift left] \ar["{0_0^A}", r, shift right,
      swap] & A \ar["{\id_A}", r] & A
    \end{tikzcd}
  \]
  commuta poichè \(\cat C(0,A)\) è banale (\(0\) è iniziale). Sia
  \(j:A \to Q\) tale che
  \[
    \begin{tikzcd}
      0 \ar["{\exists!}", r, shift left] \ar["{0_0^A}", r, shift right,
      swap] & A \ar["{j}", r] & Q
    \end{tikzcd}
  \]
  commuta. Esiste uno e una sola freccia \(A \to Q\) che fa commutare
  \[
    \begin{tikzcd}
      A \ar["{\id_A}", r] \ar["j", dr, swap] & A \ar[d] \\
      & q
    \end{tikzcd}
  \]
  ed è \(j\) stessa.
\end{proof}

Abbiamo visto che in un biprodotto le iniezioni
\(i_A : A \to A \oplus B\) e \(i_B : B \to A \oplus B\) sono
monomorfismi, mentre le proiezioni \(p_A : A \oplus B \to A\) e
\(p_B : A \oplus B \to B\) sono epimorfismi. Per di più, a causa della
Definizione~\ref{definition:Biprodotto} si ha che
\(p_B i_A = 0_A^B\) e \(p_A i_B = 0_B^A\). E quindi è lecito
chiedersi se anche in questo caso si può avere qualche coppia ker-coker.

\begin{lemma}\label{lemma:KerCokerInjProj}
  Consideriamo in una categoria additiva \(\cat C\) il biprodotto
  \[
    \begin{tikzcd}
      A \ar["{i_A}", r, shift right, swap] & A \oplus B \ar["{p_A}", l,
      shift right, swap] \ar["{p_B}", r, shift left] & B \ar["{i_B}", l,
      shift left]
    \end{tikzcd}
  \]
  Allora
  \[
    \begin{tikzcd}[row sep=tiny]
      A \ar["{i_A}", r] & A \oplus B \ar["{p_B}", r] & B \\
      B \ar["{i_B}", r] & A \oplus B \ar["{p_A}", r] & A
    \end{tikzcd}
  \]
  sono coppie ker-coker.
\end{lemma}

In effetti, sempre guardando a \(\Modu_R\), abbiamo che
\(\ker \pi_A = A \oplus 0 \cong A\) mentre
\(\coker i_A = \frac{A \oplus B}{\im i_A} = \frac{A \oplus B}{A \oplus
  0} \cong B\).

\begin{proof}
  Facciamo la dimostrazione solo la prima coppia, la dimostrazione
  dell'altra è simile. Il diagramma
  \[
    \begin{tikzcd}
      A \ar["{i_A}", r] & A \oplus B \ar["{p_B}", r, shift left]
      \ar["{0_{A \oplus B}^B}", r, shift right, swap] & B
    \end{tikzcd}
  \]
  commuta, vedi Definizione~\ref{definition:Biprodotto}. Consideriamo
  ora il diagramma commutativo
  \[
    \begin{tikzcd}
      C \ar["{j}", r] & A \oplus B \ar["{p_p}", r, shift left]
      \ar["{0_{A \oplus B}^B}", r, shift right, swap] & B
    \end{tikzcd}
  \]
  Essendo \(i_A\) un monomorfismo, allora è sufficiente mostrare
  costruire un morfismo \(C \to A\) che fa commutare
  \[
    \begin{tikzcd}
      A \ar["{i_A}",r] & A \oplus B \\
      C \ar[u] \ar["{j}", ur, swap]
    \end{tikzcd}
  \]
  Ciò che possiamo prendere in esame con quello che abbiamo è
  \[
    p_A j : C \functo{j} A \oplus B \functo{p_A} A
  \]
  Infatti
  \begin{align*}
    j &= \id_{A \oplus B} j \underbrace{=}_{\text{\eqref{eqn:Biprod3}}}
        (i_A p_A + i_B p_B) j = \\
      &= i_A p_A j + i_B \underbrace{p_B j}_{\mathclap{\cramped{= 0_{A \oplus B}^B j = 0_C^B}}} = i_A p_A j . \qedhere
  \end{align*}
\end{proof}

In \(\Modu_R\), sapresti calcolare il kernel di
\(g \oplus g' : N \oplus N' \to R \oplus R'\)? La risposta è abbastanza
semplice:
\[
  \ker (g \oplus g') = \ker g \oplus \ker g' .
\]
Nemmeno il calcolo di \(\coker(f \oplus f')\) con
\(f \oplus f' : M \oplus M' \to N \oplus N'\) nasconde particolari
difficoltà
\begin{align*}
  \coker (f \oplus f') &= \frac{N \oplus N'}{\im (f \oplus f')} = \frac{N \oplus N'}{\im f \oplus
                         \im f'} \cong \\
                       &\cong\frac{N}{\im f} \oplus \frac{N'}{\im f'} = \coker f \oplus \coker f' . 
\end{align*}
È l'intuizione che ci basta per esportare questo discorso ad un contesto
più generale. Ecco come suona:

\begin{lemma}\label{lemma:KerCokerOplus}
  Siano in una categoria additiva \(\cat C\)
  \[
    \begin{tikzcd}[row sep=tiny, column sep=small]
      A \ar["{f}", r] & B \ar["{g}", r] & C \\
      A' \ar["{f'}", r] & B' \ar["{g'}", r] & C'
    \end{tikzcd}
  \]
  con \(f\) e \(f'\) kernel di \(g\) e \(g'\) rispettivamente.  Allora
  \(f \oplus f' : A \oplus A' \to B \oplus B'\) è kernel di
  \(g \oplus g' : B \oplus B' \to C \oplus C'\). Dualizzando, si ha che se
  \(g\) e \(g'\) sono cokernel di \(f\) e \(f'\) rispettivamente, allora
  \(g \oplus g'\) è cokernel di \(f \oplus f'\). In particolare, se i
  diagrammi qui sopra sono coppie ker-coker, allora anche
  \[
    \begin{tikzcd}
      A \oplus A' \ar["{f \oplus f'}", r] & B \oplus B' \ar["{g \oplus
        g'}", r] & C \oplus C'
    \end{tikzcd}
  \]
  lo è.
\end{lemma}

Quindi ``la somma di due coppie ker-coker è una coppia ker-coker.''

\begin{proof}
  Verifichiamo che
  \[
    \begin{tikzcd}
      A \oplus A' \ar["{f \oplus f'}", r] & B \oplus B' \ar["{g \oplus
        g'}", r, shift left] \ar["{0_{B \oplus B'}^{C \oplus C'}}", r,
      shift right, swap] & C \oplus C'
    \end{tikzcd}
  \]
  commuta. Poiché \(\oplus : \cat C \times \cat C \to \cat C\) è un
  bifuntore {\color{red} [scrivere di questa cosa nell'introduzione]},
  infatti
  \[
    (g \oplus g') (f \oplus f') = (g f) \oplus (g' f)
    = 0_{A \oplus A'}^{C \oplus C'}
  \]
  Consideriamo ora un diagramma commutativo
  \[
    \begin{tikzcd}
      D \ar["{h}", r] & B \oplus B' \ar["{g \oplus g'}", r, shift left]
      \ar["{0_{B \oplus B'}^{C \oplus C'}}", r, shift right, swap] & C
      \oplus C'
    \end{tikzcd}
  \]
  e troviamo un modo di costruire una freccia \(D \to A \oplus A'\) in
  modo che commuti
  \[
    \begin{tikzcd}
      A \oplus A' \ar["{f \oplus f'}", r] & B \oplus B' \\
      D \ar["{h}", ur, swap] \ar["{}", u]
    \end{tikzcd}
  \]
  Abbiamo le frecce
  \begin{align*}
    & p_B^{BB'} h : D \to B \\
    & p_{B'}^{BB'} h : D \to B'
  \end{align*}
  che rendono commutativi i diagrammi
  \[
    \begin{tikzcd}
      D \ar["{p_B^{BB'} h}", r] & B \ar["{g}", r, shift left]
      \ar["{0_B^C}", r, shift right, swap] & C
    \end{tikzcd}
  \]
  e
  \[
    \begin{tikzcd}
      D \ar["{p_{B'}^{BB'} h}", r] & B' \ar["{g'}", r, shift left]
      \ar["{0_{B'}^{C'}}", r, shift right, swap] & C'
    \end{tikzcd}
  \]
  rispettivamente. Per la proprietà universale di equalizzatore,
  esistono unici \(i : D \to A\) e \(j : D \to A'\) tali che
  \begin{align*}
    & f i = p_B^{BB'} h \\
    & f' j = p_{B'}^{BB'} h .
  \end{align*}
  La proprietà universale di prodotto ci permette di introdurre
  \((i,j) : D \to A \oplus A'\).  Calcoliamo ora:
  \begin{align*}
    & p_B^{BB'} (f \oplus f') (i,j) = f p_A^{AA'} (i,j) = f i = p_B^{BB'} h \\
    & p_{B'}^{BB'} (f \oplus f') (i,j) = f' p_{A'}^{AA'} (i,j) = f' j = p_{B'}^{BB'} h
  \end{align*}
  Quindi, sempre per la proprietà universale di prodotto, possiamo
  concludere che
  \[
    (f \oplus f') (i,j) = h.
  \]
  La parte dell'unicità della freccia \(D \to A \oplus A'\) è immediata:
  \(f \oplus f'\) è un monomorfismo essendo \(f\) oppure \(f'\) -- in
  questo caso entrambi -- monici (ricorda che i kernel sono
  equalizzatori e quindi sono monomorfismi).
\end{proof}



\section{Stutture e categorie esatte}

\begin{definition}\label{definition:CategorieEsatte}
  Una {\em struttura esatta} per una categoria additiva \(\cat C\) è una
  classe \(\cat E\) di coppie ker-coker in \(\cat C\) con le seguenti
  proprietà:
  \begin{enumerate}[leftmargin=*, label=(E\arabic*),
    ref=Definizione~\ref{definition:CategorieEsatte}-E\arabic*]
  \item \label{item:EO} Per ogni \(a \in \objcat C\), le coppie
    ker-coker
    \[
      \begin{tikzcd}[column sep=small]
        A \ar["{\id_A}", r] & A \ar["{\ast}", r] & \ast
      \end{tikzcd}
    \]
    stanno in \(\cat E\).
  \item \label{item:EOop} Per ogni \(a \in \objcat C\), le coppie
    ker-coker
    \[
      \begin{tikzcd}[column sep=small]
        \ast \ar["{\ast}", r] & A \ar["{\id_A}", r] & A
      \end{tikzcd}
    \]
    stanno in \(\cat E\).
  \item \label{item:E1} Se \(\cat E\) contiene le coppie ker-coker
    \[
      \begin{tikzcd}[column sep=small, row sep=tiny]
        A \ar["{f}", r] & B \ar["{\ast}", r] & \ast \\
        B \ar["{g}", r] & C \ar["{\ast}", r] & \ast
      \end{tikzcd}
    \]
    allora contiene anche le coppie ker-coker
    \[
      \begin{tikzcd}[column sep=small]
        A \ar["{g \circ f}", r] & C \ar["{\ast}", r] & \ast
      \end{tikzcd}
    \]
  \item \label{item:E1op} Se \(\cat E\) contiene le coppie ker-coker
    \[
      \begin{tikzcd}[column sep=small, row sep=tiny]
        \ast \ar["{\ast}", r] & A \ar["{f}", r] & B \\
        \ast \ar["{\ast}", r] & B \ar["{g}", r] & C
      \end{tikzcd}
    \]
    allora anche le coppie ker-coker
    \[
      \begin{tikzcd}[column sep=small]
        \ast \ar["{\ast}", r] & A \ar["{g \circ f}", r] & C
      \end{tikzcd}
    \]
    stanno in \(\cat E\).
  \item \label{item:E2} Se
    \[
      \begin{tikzcd}[column sep=small]
        A \ar["{i}", r] & B \ar["{\ast}", r] & \ast
      \end{tikzcd}
    \]
    sta in \(\cat E\) e \(f : A \to A'\) è in \(\cat C\), il pushout di
    \(i\) lungo \(f\) esiste in \(\cat C\) e, indicandolo con
    \(i' : A' \to B'\), anche le coppie ker-coker
    \[
      \begin{tikzcd}[column sep=small]
        A' \ar["{i'}", r] & B' \ar["{\ast}", r] & \ast
      \end{tikzcd}
    \]
    stanno in \(\cat E\).
  \item \label{item:E2op} Se
    \[
      \begin{tikzcd}[column sep=small]
        \ast \ar["{\ast}", r] & A \ar["{p}", r] & B
      \end{tikzcd}
    \]
    sta in \(\cat E\) e \(f : C' \to C\) è in \(\cat C\), il pullback di
    \(p\) lungo \(f\) esiste in \(\cat C\) e, indicandolo con
    \(p' : B' \to C'\), anche le coppie ker-coker
    \[
      \begin{tikzcd}[column sep=small]
        \ast \ar["{\ast}", r] & A' \ar["{p'}", r] & B'
      \end{tikzcd}
    \]
    stanno in \(\cat E\).
  \end{enumerate}
  Una {\em categoria esatta} è una coppia \((\cat C,\cat E)\) come sopra
  e gli elementi di \(\cat E\) si chiamano {\em sequenze esatte corte}.
\end{definition}

Primi esempi di sequenze esatte corte derivano coinvolgono le identità.

\begin{proposition}\label{proposition:ExactId}
  Sia \((\cat C,\cat E)\) una categoria esatta. Sono successioni esatte
  corte
  \[
    \begin{tikzcd}[row sep=tiny, column sep=small]
      0 \ar["{\exists!}", r] & A \ar["{\id_A}", r] & A \\
      A \ar["{\id_A}", r] & A \ar["{\exists!}", r] & 0
    \end{tikzcd}
  \]
\end{proposition}

% In \(\Modu_R\) questo vuol dire una cosa piuttosto semplice: il kernel
% dell'omomorfismo \(\id_M : M \to M\) è banale, mentre il cokernel è
% \(M\) stesso.

\begin{proof}
  Grazie al Lemma~\ref{lemma:KerCokerId}, le due coppie sono
  ker-coker. Poi si invocano~\ref{item:EO} e~\ref{item:EOop}.
\end{proof}

\begin{recall}
  In generale, in una categoria \(\cat C\) con oggetto terminale \(1\)
  un pullback di
  \[
    \begin{tikzcd}
      & A \ar["{\exists!}",d] \\
      B \ar["{\exists!}", r, swap] & 1
    \end{tikzcd}
  \]
  è un prodotto di \(A\) e \(B\). Dualmente, se \(\cat C\) è una
  categoria con oggetto iniziale \(0\), allora un pushout di
  \[
    \begin{tikzcd}
      0 \ar["{\exists!}",r] \ar["{\exists!}", d, swap]  & A \\
      B
    \end{tikzcd}
  \]
  è un coprodotto di \(A\) e \(B\).
\end{recall}

\begin{proposition}\label{proposition:ExactInjProj}
  Sia \((\cat C,\cat E)\) una categoria esatta e sia
  \[
    \begin{tikzcd}
      A \ar["{i_A}", r, shift right, swap] & A \oplus B \ar["{p_A}", l,
      shift right, swap] \ar["{p_B}", r, shift left] & B \ar["{i_B}", l,
      shift left]
    \end{tikzcd}
  \]
  un biprodotto. Allora
  \[
    \begin{tikzcd}[row sep=tiny]
      A \ar["{i_A}", r] & A \oplus B \ar["{p_B}", r] & B \\
      B \ar["{i_B}", r] & A \oplus B \ar["{p_A}", r] & A
    \end{tikzcd}
  \]
  sono sequenze esatte corte, cioè coppie ker-coker in \(\cat E\).
\end{proposition}

Il Lemma~\ref{lemma:KerCokerInjProj} ci dava queste due coppie
ker-coker. Questo proposizione invece dice che ogni struttura esatta
contiene queste coppie ker-coker.

\begin{proof}
  Per il richiamo appena fatto,
  \[
    \begin{tikzcd}
      A \oplus B \ar["{p_A}",r] \ar["{p_B}",d,swap] & A \ar["{\exists!}",d] \\
      B \ar["{\exists!}", r, swap] & 0
    \end{tikzcd}
  \]
  è un quadrato di pushout. Inoltre per la
  Proposizione~\ref{proposition:ExactId}, la base di questo diagramma è
  parte della sequenza esatta corta \begin{tikzcd}[cramped, column
    sep=small] B \ar["{\id_B}", r] & B \ar["{\exists!}", r] & 0
  \end{tikzcd}. Possiamo appellarci a \ref{item:E2op} ora: il pullback
  di \(B \to 0\) lungo \(A \to 0\), va a dire \(p_A\), è tale che le
  coppie ker-coker
  \[
    \begin{tikzcd}[column sep=small]
      \ast \ar["{\ast}", r] & A \oplus B \ar["{p_A}", r] & A
    \end{tikzcd}
  \]
  stanno in \(\cat E\). Il Lemma~\ref{lemma:KerCokerInjProj} ci dà
  quella che ci serve.
\end{proof}

\begin{recall}
  In generale, in una categoria \(\cat C\) con oggetto terminale \(1\),
  se
  \[
    \begin{tikzcd}[column sep=small]
      A & A \times 1 \ar["{p_A}", l, swap] \ar["{p_1}", r] & 1
    \end{tikzcd}
  \]
  è un prodotto, allora l'unico morfismo \(A \to A \times 1\) che fa
  commutare
  \[
    \begin{tikzcd}
      & A \ar["{\id_A}", dl, swap] \ar["{}", d]  \ar["{\exists!}",dr] \\
      A & A \times 1 \ar["{p_A}", l] \ar["{p_1}", r, swap] & 1
    \end{tikzcd}
  \]
  è un isomorfismo. Quindi, se \(\cat C\) è anche preadditiva, allora
  questo isomorfismo è \(i_A : A \to A \oplus 0\) che ha inversa
  \(p_A : A \oplus 0 \to A\).
\end{recall}

\begin{proposition}\label{proposition:SumExactSeqsIsExact}
  ``In una categoria esatta corta la somma di sequenze esatte corte è
  esatta corta''. Vale a dire: in una categoria esatta
  \((\cat C,\cat E)\), se
  \[
    \begin{tikzcd}[row sep=tiny, column sep=small]
      A \ar["{f}", r] & B \ar["{g}", r] & C \\
      A' \ar["{f'}", r, swap] & B' \ar["{g'}", r, swap] & C'
    \end{tikzcd}
  \]
  sono sequenze esatte corte, allora anche
  \begin{equation}
    \label{cd:SumExactSeqs}
    % \begin{tikzcd}[row sep=small]
    %   A \oplus B \ar["{f \oplus g}", r] & A' \oplus B' \ar["{f' \oplus
    %     g'}", r] & A'' \oplus B''
    % \end{tikzcd}
    \begin{tikzcd}[row sep=small]
      A \oplus A' \ar["{f \oplus g}", r] & B \oplus B' \ar["{f' \oplus
        g'}", r] & C \oplus C'
    \end{tikzcd}
  \end{equation}
  lo è.
\end{proposition}

Quindi \(\cat E\) è chiusa rispetto alla somma di sequenze esatte corte.

\begin{proof}
  Il piano per la dimostrazione si struttura su questa osservazione: se
  riusciamo a dimostrare che
  \begin{equation}
    \begin{tikzcd}[row sep=tiny]
      A \oplus A' \ar["{f \oplus \id_{A'}}", r] & B \oplus A' \ar["{f' p_{A'}^{BA'}}",
      r] & C \\
      B \oplus A' \ar["{\id_B \oplus g}", r, swap] & B \oplus B' \ar["{g'
        p_{B'}^{BB'}}", r, swap] & C'
    \end{tikzcd}
    \label{cd:SumExactSeqsPieces}
  \end{equation}
  sono esatte corte allora, a causa di~\ref{item:E1op} sono sequenze
  esatte le coppie ker-coker
  \[
    \begin{tikzcd}[column sep=small, row sep=tiny]
      A \oplus A' \ar["{f \oplus g}", r] & B \oplus B' \ar["\ast",
      r] & \ast \\
    \end{tikzcd}
  \]
  A questo punto Lemma~\ref{lemma:KerCokerOplus} permette di concludere
  l'opera.\newline Mostriamo solo che la prima
  delle~\eqref{cd:SumExactSeqsPieces} è esatta corta, perché l'altra si
  fa similmente. A causa del Lemma~\ref{lemma:KerCokerInjProj} e
  della~\ref{item:E1op}, è sufficiente dimostrare che la prima
  delle~\ref{cd:SumExactSeqsPieces} è una coppia ker-coker. Il
  Lemma~\ref{lemma:KerCokerOplus} ci dà una coppia ker-coker che
  coinvolge \(f \oplus \id_{A'}\):
  \[
    \begin{tikzcd}
      A \oplus A' \ar["{f \oplus \id_{A'}}", r] & B \oplus A' \ar["{g \oplus
        0_{A'}^0}", r] & C \oplus 0
    \end{tikzcd}
  \]
  Ricordando ora che \(p_{C}^{C0} : C \oplus 0 \to C\) è un isomorfismo e che
  \(p_{C}^{C0} \left( g \oplus 0_{A'}^0 \right) = g p_{B}^{BA'}\), abbiamo
  finito.
\end{proof}

Adesso riprendiamo un discorso che avevamo iniziato nella
Proposizione~\ref{proposition:FromPullbackToKernel}.


%%% Local Variables:
%%% mode: LaTeX
%%% TeX-master: "../main"
%%% TeX-engine: luatex
%%% ispell-local-dictionary: "italian"
%%% End:
