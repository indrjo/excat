
\chapter{Categorie esatte}

In questo capitolo raccogliamo una serie di risultati elementari sulle
successioni esatte corte, seguendo da vicino la trattazione che si può
trovare in testi {\em elementari} come~\cite{buehler:exactcategories}.



\section{Coppie kernel-cokernel}

\begin{definition}\label{definition:KerCoker}
  Una {\em coppia kernel-cokernel} in una categoria additiva \(\cat C\)
  consiste di due morfismi consecutivi
  \[
    \begin{tikzcd}[column sep=small]
      A \ar["i", r] & B \ar["p", r] & C
    \end{tikzcd}
  \]
  tali che \(i\) è un kernel di \(p\) e \(p\) è un cokernel di \(i\).
\end{definition}

\begin{example}
  Sia in \(\Modu_R\) un omomorfismo \(f : M \to N\) e consideriamo la
  proiezione al quoziente
  \[
    \pi : M \to M/\ker f .
  \]
  Come si è visto (Esempio~\ref{example:KernelInModR}) il kernel è
  \(\ker \pi = \ker f\) con l'inclusione. Viceversa
  \[
    i : \ker f \hookrightarrow M
  \]
  ha come cokernel \(\pi\) (Esempio~\ref{example:CokernelInModR}). Quindi
  \[
    \begin{tikzcd}[column sep=small]
      \ker f \ar["i", hookrightarrow, r] & M \ar["\pi", r] & \frac{M}{\ker
        f}
    \end{tikzcd}
  \]
  è una coppia ker-coker in \(\Modu_R\). In generale, se \(H\) è un
  sottomodulo di \(M\), allora
  \[
    \begin{tikzcd}[column sep=small]
      H \ar[hookrightarrow, r] & M \ar["{\pi_H}", r] & \frac{M}{H}
    \end{tikzcd}
  \]
  è una coppia ker-coker.
\end{example}

I seguenti lemmi sono in preparazione alla successiva definizione. Sono
fatti la cui evidenza è sbalorditiva quando si tiene a mente una
categoria additiva come \(\Modu_R\).

Qual è il kernel dell'omomorfismo \(\id_M : M \to M\)? Il modulo banale. E
il cokernel di \(0 \to M\)? È \(\frac{M}{\{0\}} \cong M\). Il kernel di
\(M \to 0\)? \(M\) stesso. Il cokernel di \(\id_M : M \to M\)?
\(\frac{M}{M} \cong 0\). Ecco come si scrive in un contesto astratto come
questo.

\begin{lemma}\label{lemma:KerCokerId}
  In una categoria additiva \(\cat C \),
  \[
    \begin{tikzcd}[column sep=small]
      0 \ar["{\exists!}", r] & A \ar["{\id_A}", r] & A
    \end{tikzcd}
  \]
  è una coppia ker-coker. Dualmente,
  \[
    \begin{tikzcd}[column sep=small]
      A \ar["{\id_A}", r] & A \ar["{\exists!}", r] & 0
    \end{tikzcd}
  \]
  è una coppia ker-coker
\end{lemma}

\begin{proof}
  Il diagramma
  \[
    \begin{tikzcd}
      0 \ar["{\exists!}", r] & A \ar["{\id_A}", r, shift left] \ar["{0_A^A}",
      r, shift right, swap] & A
    \end{tikzcd}
  \]
  commuta perché \(\cat C(0,A)\) è banale, essendo \(0\) iniziale. Sia
  \(i : E \to A\) tale che
  \[
    \begin{tikzcd}
      E \ar["{i}", r] & A \ar["{\id_A}", r, shift left] \ar["{0_A^A}",
      r, shift right, swap] & A
    \end{tikzcd}
  \]
  commuta, cioè \(i = 0_E^A \). Essendo 0 terminale, c'è esattamente un
  \(E \to 0\). Il triangolo
  \[
    \begin{tikzcd}
      0 \ar["{\exists!}", r] & A \\
      E \ar["{\exists!}", u] \ar["i", ur, swap]
    \end{tikzcd}
  \]
  commuta per come sono definiti i morfismi nulli.  Verifichiamo ora che
  \(\id_A\) è cokernel di \(0 \to A\). Anzitutto
  \[
    \begin{tikzcd}
      0 \ar["{\exists!}", r, shift left] \ar["{0_0^A}", r, shift right,
      swap] & A \ar["{\id_A}", r] & A
    \end{tikzcd}
  \]
  commuta poichè \(\cat C(0,A)\) è banale (\(0\) è iniziale). Sia
  \(j:A \to Q\) tale che
  \[
    \begin{tikzcd}
      0 \ar["{\exists!}", r, shift left] \ar["{0_0^A}", r, shift right,
      swap] & A \ar["{j}", r] & Q
    \end{tikzcd}
  \]
  commuta. Esiste uno e una sola freccia \(A \to Q\) che fa commutare
  \[
    \begin{tikzcd}
      A \ar["{\id_A}", r] \ar["j", dr, swap] & A \ar[d] \\
      & q
    \end{tikzcd}
  \]
  ed è \(j\) stessa.
\end{proof}

Abbiamo visto che in un biprodotto le iniezioni
\(i_A : A \to A \oplus B\) e \(i_B : B \to A \oplus B\) sono
monomorfismi, mentre le proiezioni \(p_A : A \oplus B \to A\) e
\(p_B : A \oplus B \to B\) sono epimorfismi. Per di più, a causa della
Definizione~\ref{definition:Biprodotto} si ha che
\(p_B i_A = 0_A^B\) e \(p_A i_B = 0_B^A\). E quindi è lecito
chiedersi se anche in questo caso si può avere qualche coppia ker-coker.

\begin{lemma}\label{lemma:KerCokerInjProj}
  Consideriamo in una categoria additiva \(\cat C\) il biprodotto
  \[
    \begin{tikzcd}
      A \ar["{i_A}", r, shift right, swap] & A \oplus B \ar["{p_A}", l,
      shift right, swap] \ar["{p_B}", r, shift left] & B \ar["{i_B}", l,
      shift left]
    \end{tikzcd}
  \]
  Allora
  \[
    \begin{tikzcd}[row sep=tiny]
      A \ar["{i_A}", r] & A \oplus B \ar["{p_B}", r] & B \\
      B \ar["{i_B}", r] & A \oplus B \ar["{p_A}", r] & A
    \end{tikzcd}
  \]
  sono coppie ker-coker.
\end{lemma}

In effetti, sempre guardando a \(\Modu_R\), abbiamo che
\(\ker \pi_A = A \oplus 0 \cong A\) mentre
\(\coker i_A = \frac{A \oplus B}{\im i_A} = \frac{A \oplus B}{A \oplus
  0} \cong B\).

\begin{proof}
  Facciamo la dimostrazione solo la prima coppia, la dimostrazione
  dell'altra è simile. Il diagramma
  \[
    \begin{tikzcd}
      A \ar["{i_A}", r] & A \oplus B \ar["{p_B}", r, shift left]
      \ar["{0_{A \oplus B}^B}", r, shift right, swap] & B
    \end{tikzcd}
  \]
  commuta, vedi Definizione~\ref{definition:Biprodotto}. Consideriamo
  ora il diagramma commutativo
  \[
    \begin{tikzcd}
      C \ar["{j}", r] & A \oplus B \ar["{p_p}", r, shift left]
      \ar["{0_{A \oplus B}^B}", r, shift right, swap] & B
    \end{tikzcd}
  \]
  Essendo \(i_A\) un monomorfismo, allora è sufficiente mostrare
  costruire un morfismo \(C \to A\) che fa commutare
  \[
    \begin{tikzcd}
      A \ar["{i_A}",r] & A \oplus B \\
      C \ar[u] \ar["{j}", ur, swap]
    \end{tikzcd}
  \]
  Ciò che possiamo prendere in esame con quello che abbiamo è
  \[
    p_A j : C \functo{j} A \oplus B \functo{p_A} A
  \]
  Infatti
  \begin{align*}
    j &= \id_{A \oplus B} j \underbrace{=}_{\text{\eqref{eqn:Biprod3}}}
        (i_A p_A + i_B p_B) j = \\
      &= i_A p_A j + i_B \underbrace{p_B j}_{\mathclap{\cramped{= 0_{A \oplus B}^B j = 0_C^B}}} = i_A p_A j . \qedhere
  \end{align*}
\end{proof}

In \(\Modu_R\), sapresti calcolare il kernel di
\(g \oplus g' : N \oplus N' \to R \oplus R'\)? La risposta è abbastanza
semplice:
\[
  \ker (g \oplus g') = \ker g \oplus \ker g' .
\]
Nemmeno il calcolo di \(\coker(f \oplus f')\) con
\(f \oplus f' : M \oplus M' \to N \oplus N'\) nasconde particolari
difficoltà
\begin{align*}
  \coker (f \oplus f') &= \frac{N \oplus N'}{\im (f \oplus f')} = \frac{N \oplus N'}{\im f \oplus
                         \im f'} \cong \\
                       &\cong\frac{N}{\im f} \oplus \frac{N'}{\im f'} = \coker f \oplus \coker f' . 
\end{align*}
È l'intuizione che ci basta per esportare questo discorso ad un contesto
più generale. Ecco come suona:

\begin{lemma}\label{lemma:KerCokerOplus}
  Siano in una categoria additiva \(\cat C\)
  \[
    \begin{tikzcd}[row sep=tiny, column sep=small]
      A \ar["{f}", r] & B \ar["{g}", r] & C \\
      A' \ar["{f'}", r] & B' \ar["{g'}", r] & C'
    \end{tikzcd}
  \]
  con \(f\) e \(f'\) kernel di \(g\) e \(g'\) rispettivamente.  Allora
  \(f \oplus f' : A \oplus A' \to B \oplus B'\) è kernel di
  \(g \oplus g' : B \oplus B' \to C \oplus C'\). Dualizzando, si ha che se
  \(g\) e \(g'\) sono cokernel di \(f\) e \(f'\) rispettivamente, allora
  \(g \oplus g'\) è cokernel di \(f \oplus f'\). In particolare, se i
  diagrammi qui sopra sono coppie ker-coker, allora anche
  \[
    \begin{tikzcd}
      A \oplus A' \ar["{f \oplus f'}", r] & B \oplus B' \ar["{g \oplus
        g'}", r] & C \oplus C'
    \end{tikzcd}
  \]
  lo è.
\end{lemma}

Quindi ``la somma di due coppie ker-coker è una coppia ker-coker.''

\begin{proof}
  Verifichiamo che
  \[
    \begin{tikzcd}
      A \oplus A' \ar["{f \oplus f'}", r] & B \oplus B' \ar["{g \oplus
        g'}", r, shift left] \ar["{0_{B \oplus B'}^{C \oplus C'}}", r,
      shift right, swap] & C \oplus C'
    \end{tikzcd}
  \]
  commuta. Poiché \(\oplus : \cat C \times \cat C \to \cat C\) è un
  bifuntore {\color{red} [scrivere di questa cosa nell'introduzione]},
  infatti
  \[
    (g \oplus g') (f \oplus f') = (g f) \oplus (g' f)
    = 0_{A \oplus A'}^{C \oplus C'}
  \]
  Consideriamo ora un diagramma commutativo
  \[
    \begin{tikzcd}
      D \ar["{h}", r] & B \oplus B' \ar["{g \oplus g'}", r, shift left]
      \ar["{0_{B \oplus B'}^{C \oplus C'}}", r, shift right, swap] & C
      \oplus C'
    \end{tikzcd}
  \]
  e troviamo un modo di costruire una freccia \(D \to A \oplus A'\) in
  modo che commuti
  \[
    \begin{tikzcd}
      A \oplus A' \ar["{f \oplus f'}", r] & B \oplus B' \\
      D \ar["{h}", ur, swap] \ar["{}", u]
    \end{tikzcd}
  \]
  Abbiamo le frecce
  \begin{align*}
    & p_B^{BB'} h : D \to B \\
    & p_{B'}^{BB'} h : D \to B'
  \end{align*}
  che rendono commutativi i diagrammi
  \[
    \begin{tikzcd}
      D \ar["{p_B^{BB'} h}", r] & B \ar["{g}", r, shift left]
      \ar["{0_B^C}", r, shift right, swap] & C
    \end{tikzcd}
  \]
  e
  \[
    \begin{tikzcd}
      D \ar["{p_{B'}^{BB'} h}", r] & B' \ar["{g'}", r, shift left]
      \ar["{0_{B'}^{C'}}", r, shift right, swap] & C'
    \end{tikzcd}
  \]
  rispettivamente. Per la proprietà universale di equalizzatore,
  esistono unici \(i : D \to A\) e \(j : D \to A'\) tali che
  \begin{align*}
    & f i = p_B^{BB'} h \\
    & f' j = p_{B'}^{BB'} h .
  \end{align*}
  La proprietà universale di prodotto ci permette di introdurre
  \((i,j) : D \to A \oplus A'\).  Calcoliamo ora:
  \begin{align*}
    & p_B^{BB'} (f \oplus f') (i,j) = f p_A^{AA'} (i,j) = f i = p_B^{BB'} h \\
    & p_{B'}^{BB'} (f \oplus f') (i,j) = f' p_{A'}^{AA'} (i,j) = f' j = p_{B'}^{BB'} h
  \end{align*}
  Quindi, sempre per la proprietà universale di prodotto, possiamo
  concludere che
  \[
    (f \oplus f') (i,j) = h.
  \]
  La parte dell'unicità della freccia \(D \to A \oplus A'\) è immediata:
  \(f \oplus f'\) è un monomorfismo essendo \(f\) oppure \(f'\) -- in
  questo caso entrambi -- monici (ricorda che i kernel sono
  equalizzatori e quindi sono monomorfismi).
\end{proof}



\section{Stutture e categorie esatte}

\begin{recall}
  Sia \(\cat C\) una categoria. Definiamo \(\cat C^{\to\to}\) la
  categoria in cui
  \begin{itemize}
  \item Gli oggetti sono coppie di frecce
    \(\begin{tikzcd}[column sep=small, cramped] A \ar["{f}", r] & B
      \ar["{g}", r] & C \end{tikzcd}\) di \(\cat C\).
  \item I morfismi da
    \(\begin{tikzcd}[column sep=small, cramped] A \ar["{f}", r] & B
      \ar["{g}", r] & C \end{tikzcd}\) a
    \(\begin{tikzcd}[column sep=small, cramped] A' \ar["{f'}", r] & B'
      \ar["{g'}", r] & C' \end{tikzcd}\) sono triple di frecce di
    \(\cat C\)
    \[
      \left(
        \begin{tikzcd}[row sep=small, cramped] A \ar["{\alpha}", d, swap] \\
          A' \end{tikzcd},
        \begin{tikzcd}[row sep=small, cramped] B \ar["{\beta}", d, swap] \\
          B' \end{tikzcd},
        \begin{tikzcd}[row sep=small, cramped] C \ar["{\gamma}", d, swap] \\
          C' \end{tikzcd} \right)
    \]
    per cui commuta
    \[
      \begin{tikzcd}
        A \ar["{f}", r] \ar["{\alpha}", d, swap] & B \ar["{g}", r] \ar["{\beta}",
        d, swap] & C \ar["{\gamma}", d, swap] \\
        A' \ar["{f'}", r, swap] & B' \ar["{g'}", r, swap] & C'
      \end{tikzcd}
    \]
  \item La composizione avviene per componenti.
  \end{itemize}
\end{recall}

\begin{definition}\label{definition:CategorieEsatte}
  Una {\em struttura esatta} per una categoria additiva \(\cat C\) è una
  classe \(\cat E\) di coppie ker-coker di \(\cat C\) in cui:
  \begin{itemize}[leftmargin=*]
  \item Chiamiamo {\em monomorfismi ammissibili} le frecce
    \(i : A \to B\) per le quali esiste \(p : B \to C\) tale che
    \(\begin{tikzcd}[column sep=small, cramped] A \ar["{i}", r] & B
      \ar["{p}", r] & C \end{tikzcd}\) appartiene ad \(\cat E\). Useremo
    \(\mono\) al posto di \(\to\) per indicare monomomorfismi ammissibili
    nei diagrammi commutativi.
  \item Chiamiamo {\em epimorfismi ammissibili} le frecce
    \(p : B \to C\) per le quali esiste \(i : A \to B\) tale che
    \(\begin{tikzcd}[column sep=small, cramped] A \ar["{i}", r] & B
      \ar["{p}", r] & C \end{tikzcd}\) appartiene ad \(\cat E\). Useremo
    \(\epi\) al posto di \(\to\) per indicare epimomorfismi ammissibili
    nei diagrammi commutativi.
  \end{itemize}
  che rispetta le seguenti proprietà:
  \begin{enumerate}[leftmargin=*, label=(E\arabic*),
    % ref=Definizione~\ref{definition:CategorieEsatte}-E\arabic*
    ref=(E\arabic*)]
  \item \(\cat E\) è chiusa per isomorfismo, cioè se
    \(\begin{tikzcd}[column sep=small, cramped] A \ar["{i}", r] & B
      \ar["{p}", r] & C \end{tikzcd}\) è in \(\cat E\) ed è isomorfo a
    \(\begin{tikzcd}[column sep=small, cramped] A' \ar["{i'}", r] & B'
      \ar["{p'}", r] & C' \end{tikzcd}\) come oggetto di
    \(\cat C^{\to\to}\), allora anche quest'ultima coppia di frecce è in
    \(\cat E\).
  \item \label{item:EO} \(\id_A : A \to A\) è un monomorfismo ammissibile
    per ogni \(A \in \obj{\cat C}\).
  \item \label{item:EOop} \(\id_A : A \to A\) è un epimorfismo ammissibile
    per ogni \(A \in \obj{\cat C}\).
  \item \label{item:E1} La composizione di due monomorfismi ammissibili
    è un monomorfismo ammissibile.
  \item \label{item:E1op} La composizione di due epimorfismi ammissibili
    è un epimorfismo ammissibile.
  \item \label{item:E2} Il pushout di un monomorfismo ammissibile lungo
    un qualsiasi morfismo di \(\cat C\) esiste ed è un monomorfismo
    ammissibile.
    \[
      \begin{tikzcd}
        A \ar["{f}", d, swap] \ar["{PO}" description, dr, phantom] \ar["{i}", rightarrowtail, r] & B \ar["{}", d] \\
        A' \ar["{i'}", rightarrowtail, r, swap] & B'
      \end{tikzcd}
    \]
  \item \label{item:E2op} Il pullback di un epimorfismo ammissibile
    lungo un qualsiasi morfismo di \(\cat C\) esiste ed è un epimorfismo
    ammissibile.
    \[
      \begin{tikzcd}
        B' \ar["{}", d] \ar["{PB}" description, dr, phantom] \ar["{p'}", twoheadrightarrow, r] & C' \ar["{f}", d] \\
        B \ar["{p}", twoheadrightarrow, r, swap] & C
      \end{tikzcd}
    \]
  \end{enumerate}
  Una {\em categoria esatta} è una coppia \((\cat C,\cat E)\) come sopra
  e gli elementi di \(\cat E\) si chiamano {\em successioni esatte
    corte}.
\end{definition}

% \begin{example}
%   Nota come gli assiomi~\ref{item:E2} e~\ref{item:E2op} replicano la
%   Proposizione~\ref{proposition:PushotMonoIsMono}. Infatti sia
%   \(\cat C\) una categoria abeliana. A causa della
%   Proposizione~\ref{proposition:MonoIsKerOfItsCoker}, è molto semplice
%   figurarsi una struttura esatta: si può prender la classe dei
%   monomorfismi di \(\cat C\) e attaccare a ciascuno i rispettivi
%   cokernel.
% \end{example}

\begin{proposition}
  In categorie esatte gli isomorfismi sono monomomorfismi ed epimorfismi
  ammissibili.
\end{proposition}

\begin{proof}
  Se \(f : A \to B\) è un isomorfismo, allora abbiamo un quadrato di
  pushout e uno di pullback
  \[
    \begin{tikzcd}
      B \ar["{f^{-1}}", d, swap] \ar["{PO}" description, dr, phantom] \ar["{\id_B}", r] & B \ar["{\id_B}", d] \\
      A \ar["{f}", r, swap] & B
    \end{tikzcd}
    % \]
    % è di pushout, mentre
    % \[
    \qquad
    \begin{tikzcd}
      A \ar["{\id_A}", d, swap] \ar["{PB}" description, dr, phantom] \ar["{f}", r] & B \ar["{f^{-1}}", d] \\
      A \ar["{\id_A}", r, swap] & A
    \end{tikzcd}
  \]
  Grazie a~\ref{item:EO}, \ref{item:EOop}, \ref{item:E2}
  e~\ref{item:E2op}, possiamo concludere.
\end{proof}

Primi esempi di successioni esatte corte coinvolgono le identità.

\begin{proposition}\label{proposition:ExactId}
  Sia \((\cat C,\cat E)\) una categoria esatta. Sono successioni esatte
  corte
  \[
    \begin{tikzcd}[row sep=tiny, column sep=small]
      0 \ar["{\exists!}", r] & A \ar["{\id_A}", r] & A \\
      A \ar["{\id_A}", r] & A \ar["{\exists!}", r] & 0
    \end{tikzcd}
  \]
  Cioè coppie ker-coker in \(\cat E\).
\end{proposition}

\begin{proof}
  Grazie al Lemma~\ref{lemma:KerCokerId}, le due coppie sono
  ker-coker. Poi si invocano~\ref{item:EO} e~\ref{item:EOop}.
\end{proof}

\begin{recall}
  In generale, in una categoria \(\cat C\) con oggetto terminale \(1\)
  un pullback di
  \[
    \begin{tikzcd}
      & A \ar["{\exists!}",d] \\
      B \ar["{\exists!}", r, swap] & 1
    \end{tikzcd}
  \]
  è un prodotto di \(A\) e \(B\). Dualmente, se \(\cat C\) è una
  categoria con oggetto iniziale \(0\), allora un pushout di
  \[
    \begin{tikzcd}
      0 \ar["{\exists!}",r] \ar["{\exists!}", d, swap]  & A \\
      B
    \end{tikzcd}
  \]
  è un coprodotto di \(A\) e \(B\).
\end{recall}

\begin{proposition}\label{proposition:ExactInjProj}
  Sia \((\cat C,\cat E)\) una categoria esatta e sia
  \[
    \begin{tikzcd}
      A \ar["{i_A}", r, shift right, swap] & A \oplus B \ar["{p_A}", l,
      shift right, swap] \ar["{p_B}", r, shift left] & B \ar["{i_B}", l,
      shift left]
    \end{tikzcd}
  \]
  un biprodotto. Allora
  \[
    \begin{tikzcd}[row sep=tiny]
      A \ar["{i_A}", r] & A \oplus B \ar["{p_B}", r] & B \\
      B \ar["{i_B}", r] & A \oplus B \ar["{p_A}", r] & A
    \end{tikzcd}
  \]
  sono sequenze esatte corte, cioè coppie ker-coker in \(\cat E\).
\end{proposition}

Il Lemma~\ref{lemma:KerCokerInjProj} ci dava queste due coppie
ker-coker. Questo proposizione invece dice che ogni struttura esatta
contiene queste coppie ker-coker.

\begin{proof}
  Per il richiamo appena fatto,
  \[
    \begin{tikzcd}
      A \oplus B \ar["{p_A}",r] \ar["{p_B}",d,swap] & A \ar["{\exists!}",d] \\
      B \ar["{\exists!}", r, swap] & 0
    \end{tikzcd}
  \]
  è un quadrato di pullback. Inoltre per la
  Proposizione~\ref{proposition:ExactId}, la base di questo diagramma è
  parte della successione esatta corta \begin{tikzcd}[cramped, column
    sep=small] B \ar["{\id_B}", r] & B \ar["{\exists!}", r] & 0
  \end{tikzcd}. Possiamo appellarci a \ref{item:E2op} ora: il pullback
  di \(B \to 0\) lungo \(A \to 0\), vale a dire \(p_A\), è un epimorfismo
  ammissibile. Il Lemma~\ref{lemma:KerCokerInjProj} ci dà un coppia
  ker-coker: essendo chiusa per isomorfismo, questa coppia sta in
  \(\cat E\).
\end{proof}

\begin{recall}
  In generale, in una categoria \(\cat C\) con oggetto terminale \(1\),
  se
  \[
    \begin{tikzcd}[column sep=small]
      A & A \times 1 \ar["{p_A}", l, swap] \ar["{p_1}", r] & 1
    \end{tikzcd}
  \]
  è un prodotto, allora l'unico morfismo \(A \to A \times 1\) che fa
  commutare
  \[
    \begin{tikzcd}
      & A \ar["{\id_A}", dl, swap] \ar["{}", d]  \ar["{\exists!}",dr] \\
      A & A \times 1 \ar["{p_A}", l] \ar["{p_1}", r, swap] & 1
    \end{tikzcd}
  \]
  è un isomorfismo. Quindi, se \(\cat C\) è anche preadditiva, allora
  questo isomorfismo è \(i_A : A \to A \oplus 0\) che ha inversa
  \(p_A : A \oplus 0 \to A\).
\end{recall}

\begin{proposition}\label{proposition:SumExactSeqsIsExact}
  ``In una categoria esatta la somma di successioni esatte corte è
  esatta corta''. Vale a dire: in una categoria esatta
  \((\cat C,\cat E)\), se
  \[
    \begin{tikzcd}[row sep=tiny, column sep=small]
      A \ar["{f}", r] & B \ar["{g}", r] & C \\
      A' \ar["{f'}", r, swap] & B' \ar["{g'}", r, swap] & C'
    \end{tikzcd}
  \]
  sono sequenze esatte corte, allora anche
  \begin{equation}
    \label{cd:SumExactSeqs}
    % \begin{tikzcd}[row sep=small]
    %   A \oplus B \ar["{f \oplus g}", r] & A' \oplus B' \ar["{f' \oplus
    %     g'}", r] & A'' \oplus B''
    % \end{tikzcd}
    \begin{tikzcd}[row sep=small]
      A \oplus A' \ar["{f \oplus g}", r] & B \oplus B' \ar["{f' \oplus
        g'}", r] & C \oplus C'
    \end{tikzcd}
  \end{equation}
  lo è.
\end{proposition}

Quindi \(\cat E\) è chiusa rispetto alla somma di sequenze esatte corte.

\begin{proof}
  Il piano per la dimostrazione si struttura su questa osservazione: se
  riusciamo a dimostrare che
  \begin{equation}
    \begin{tikzcd}[row sep=tiny]
      A \oplus A' \ar["{f \oplus \id_{A'}}", r] & B \oplus A' \ar["{f' p_{A'}^{BA'}}",
      r] & C \\
      B \oplus A' \ar["{\id_B \oplus g}", r, swap] & B \oplus B' \ar["{g'
        p_{B'}^{BB'}}", r, swap] & C'
    \end{tikzcd}
    \label{cd:SumExactSeqsPieces}
  \end{equation}
  sono esatte corte allora, a causa di~\ref{item:E1op} sono sequenze
  esatte le coppie ker-coker
  \[
    \begin{tikzcd}[column sep=small, row sep=tiny]
      A \oplus A' \ar["{f \oplus g}", r] & B \oplus B' \ar["\ast",
      r] & \ast \\
    \end{tikzcd}
  \]
  A questo punto Lemma~\ref{lemma:KerCokerOplus} permette di concludere
  l'opera.\newline Mostriamo solo che la prima
  delle~\eqref{cd:SumExactSeqsPieces} è esatta corta, perché l'altra si
  fa similmente. A causa del Lemma~\ref{lemma:KerCokerInjProj} e
  della~\ref{item:E1op}, è sufficiente dimostrare che la prima
  delle~\ref{cd:SumExactSeqsPieces} è una coppia ker-coker. Il
  Lemma~\ref{lemma:KerCokerOplus} ci dà una coppia ker-coker che
  coinvolge \(f \oplus \id_{A'}\):
  \[
    \begin{tikzcd}
      A \oplus A' \ar["{f \oplus \id_{A'}}", r] & B \oplus A' \ar["{g
        \oplus 0_{A'}^0}", r] & C \oplus 0
    \end{tikzcd}
  \]
  Ricordando ora che \(p_{C}^{C0} : C \oplus 0 \to C\) è un isomorfismo
  e che \(p_{C}^{C0} \left( g \oplus 0_{A'}^0 \right) = g p_{B}^{BA'}\),
  abbiamo finito.
\end{proof}

% Adesso riprendiamo un discorso che avevamo iniziato nella
% Proposizione~\ref{proposition:FromPullbackToKernel}.

\begin{proposition}\label{proposition:pushout-exact-bicart}
  Sia \((\cat C, \cat E)\) una categoria esatta e consideriamo il
  diagramma commutativo
  \[
    \begin{tikzcd}
      A \ar["{f}", d, swap] \ar["{i}", rightarrowtail, r] & B \ar["{f'}", d] \\
      A' \ar["{i'}", rightarrowtail, r, swap] & B'
    \end{tikzcd}
  \]
  in cui le frecce orizzontali sono monomorfismi ammissibili. Allora
  sono equivalenti:
  \begin{enumerate}[leftmargin=*, label=(\roman*), ref=(\roman*)]
  \item\label{item:1} Il quadrato è di pushout.
  \item\label{item:2} La successione
    \[
      \begin{tikzcd}[ampersand replacement=\&]
        A \ar["{\begin{psmallmatrix} i \\ f \end{psmallmatrix}}", r] \&
        B \oplus A' \ar["{\begin{psmallmatrix} f' &
            -i' \end{psmallmatrix}}", r] \& B'
      \end{tikzcd}
    \]
    è esatta corta.
  \item\label{item:3} Il quadrato è bicartesiano (cioè di pullback e di
    pushout).
  \item\label{item:4} Il quadrato è parte del diagramma commutativo
    \[
      \begin{tikzcd}[row sep=tiny]
        A \ar["{f}", dd, swap] \ar["{i}", rightarrowtail, r] & B
        \ar["{f'}", dd, swap]
        \ar["{p}", twoheadrightarrow, dr] \\
        & & C \\
        A' \ar["{i'}", rightarrowtail, r, swap] & B' \ar["{p'}",
        twoheadrightarrow, ur, swap]
      \end{tikzcd}
    \]
    dove \((i, p)\) e \((i', p')\) sono successioni esatte corte.
  \end{enumerate}
\end{proposition}

\begin{proof}
  \ref{item:1} \Rightarrow \ref{item:2}. La
  Proposizione~\ref{proposition:FromPullbackToKernel} ci dice che
  \(\begin{psmallmatrix} f' & -i' \end{psmallmatrix}\) è cokernel di
  \(\begin{psmallmatrix} i \\ f \end{psmallmatrix}\). Quindi il piano è
  verificare che quest'ultimo morfismo è un monomorfismo
  ammissibile. Usiamo un po' di calcolo matriciale:
  \[
    \begin{pmatrix} i \\ f \end{pmatrix} = \underbrace{\begin{pmatrix} i
        & 0_{A'}^B \\ 0_A^{A'} & \id_{A'} \end{pmatrix}}_{i \oplus
      \id_{A'}}
    \begin{pmatrix} \id_A & 0_{A'}^A \\ f & \id_{A'} \end{pmatrix}
    \underbrace{\begin{pmatrix} \id_A \\
      0_A^{A'} \end{pmatrix}}_{i_A}
\]
Come abbiamo visto \(i_A\) è monomorfismo ammissibile
(Proposizione~\ref{proposition:ExactInjProj}) così come
\(i \oplus \id_{A'}\)
(Proposizione~\ref{proposition:SumExactSeqsIsExact}). Il morfismo in
mezzo è un isomorfismo (calcola esplicitamente l'inverso) e quindi è un
monomorfismo ammissibile. In definitiva, \(\begin{psmallmatrix} i \\
  f \end{psmallmatrix}\) è ammissibile perché è composizione di
  monomorfismi ammissibili.

  \ref{item:2} \Rightarrow \ref{item:3}. Questa è ovvia, sempre a causa
  della Proposizione~\ref{proposition:FromPullbackToKernel}.

  \ref{item:3} \Rightarrow \ref{item:1}. Anche questa è ovvia.
  
  \ref{item:1} \Rightarrow \ref{item:4}. Sia \(p : B \to C\) un cokernel
  di \(i : A \to B\) e, attraverso la proprietà universale di pushout,
  sia \(p' : B' \to C\) la freccia tale che \(p = f' p'\) e
  \(0_{A'}^C = p' i'\)
  \[
    \begin{tikzcd}
      A \ar["{f}", d, swap] \ar["{i}", r] & B \ar["{f'}", d, swap]
      \ar["{p}", ddr, bend left] \\
      A' \ar["{i'}", r] \ar["{0_{A'}^C}", drr, bend right, swap] &
      B' \ar["{p'}", dr] \\
      & & C
    \end{tikzcd}
  \]
  Per come abbiamo scelto \(p\), la successione
  \(\begin{tikzcd}[column sep=small, cramped] A \ar["{i}", r] & B
    \ar["{p}", r] & C \end{tikzcd}\) è esatta corta. Per provare che
  anche
  \(\begin{tikzcd}[column sep=small, cramped] A' \ar["{i'}", r] & B'
    \ar["{p'}", r] & C \end{tikzcd}\) lo è, mostriamo che \(p'\) è
  cokernel di \(i'\). Sia \(h : B' \to X\) tale che
  \(h i' = h 0_{A'}^{B'} = 0_{A'}^X\). Allora ne segue che
  \[
    h f' i = h i' f = 0_A^X .
  \]
  Poiché \(p : B \to C\) è cokernel di \(i : A \to B\), allora
  \(h f' = k p\) per un unico morfismo \(k : C \to X\). A questo punto
  abbiamo
  \begin{align*}
    & h f' = k p = k p' f' \\
    & k p' i' = 0_{A'}^X = h i'
  \end{align*}
  che, per la proprietà universale di pushout, dà \(k p' = h\). Per
  concludere, ricordiamo che essendo \(p = p' f'\) un epimorfismo, anche
  \(p'\) lo è. Questo ci permette di dire che se \(k' : C \to X\) è tale
  che \(h = k' p'\), allora \(k = k'\).

  \ref{item:4} \Rightarrow \ref{item:2}. Il piano sarà di costruire in
  qualche modo una successione esatta corta \(A \mono D \epi B'\) in cui
  \(D \cong B \oplus A'\) e per cui commmuta
  \[
    \begin{tikzcd}[row sep=small, ampersand replacement=\&]
      \& D \ar["{\cong}" description, dd] \ar["{}", twoheadrightarrow, dr] \\
      A \ar["{}", rightarrowtail, ur] \ar["{\begin{psmallmatrix} i \\
        f \end{psmallmatrix}}", dr, swap] \& \& B' \\
    \& B \oplus A' \ar["{\begin{psmallmatrix} f' &
        -i' \end{psmallmatrix}}", ur, swap]
  \end{tikzcd}
\]
Questo basta per dimostrare ciò che ci serve visto che la classe delle
successioni esatte corte è chiusa per isomorfismi.\newline Incominciamo
con le costruzioni. Sia il quadrato di pullback
\[
  \begin{tikzcd}
    D \ar["{q}", twoheadrightarrow, d, swap] \ar["{q'}",
    twoheadrightarrow, r] & B \ar["{p}", twoheadrightarrow, d] \\
    B' \ar["{p'}", r, twoheadrightarrow, swap] & C
  \end{tikzcd}
\]
in cui osserviamo che \(q\) e \(q'\) sono epimorfismi ammissibili in
quanto pullback di epimorfismi ammissibili. Sfruttando il duale della
implicazione \ref{item:1} \Rightarrow \ref{item:4}, possiamo disegnare
\[
  \begin{tikzcd}
    & & A \ar["{j}", rightarrowtail, dl, swap] \ar["{i}", rightarrowtail, d] \\
    & D \ar["{q}", twoheadrightarrow, d] \ar["{q'}",
    twoheadrightarrow, r, swap] & B \ar["{p}", twoheadrightarrow, d] \\
    A' \ar["{j'}", rightarrowtail, ur] \ar["{i'}", rightarrowtail, r,
    swap] & B' \ar["{p'}", r, twoheadrightarrow, swap] & C
  \end{tikzcd}
\]
Ecco la successione esatta corta
\(\begin{tikzcd}[column sep=small, cramped] A \ar["{j}", rightarrowtail,
  r] & D \ar["{q}", twoheadrightarrow, r] & B' \end{tikzcd}\) che
avevamo annunciato.\newline Per la proprietà universale di pullback, sia
\(k : B \to D\) tale che \(q'k = \id_B\) e \(qk = f'\). In particolare
\(q'(\id_D-kq') = 0_D^B\). Poiché \(j'\) è kernel di \(q'\), sia
\(l : D \to A'\) il morfismo tale che \(j'l =
\id_D-kq'\). Postcomponendo ambo i membri di questa equazione una volta
per \(k\) e un'altra per \(j'\) e ricordando che \(j'\) è monico,
deriviamo che \(lk = 0_B^{A'}\) e \(lj' = \id_{A'}\). Inoltre
\(i'lj = -i'f\) (per calcolo diretto) da cui segue che \(lj = -f\)
perché \(i'\) e monico.\newline Si può mostrare finalmente che
\[
  \begin{pmatrix} k & j' \end{pmatrix} : B \oplus A' \to D
  \quad \begin{pmatrix} q' \\ l \end{pmatrix} : D \to B \oplus A'
\]
sono una l'inversa dell'altra.
\end{proof}

\begin{recall}[Il lemma dei pullback]\label{recall:PBLemma}
  In una categoria \(\cat C\) consideriamo il diagramma
  \[
    \begin{tikzcd}
      \bullet \ar["{}", d] \ar["{Q_1}" description, phantom, dr]
      \ar["{}", r] & \bullet \ar["{}", d] \ar["{Q_2}" description,
      phantom, dr] \ar["{}", r] & \bullet
      \ar["{}", d] \\
      \bullet \ar["{}", r] & \bullet \ar["{}", r] & \bullet
    \end{tikzcd}
  \]
  Allora:
  \begin{enumerate}
  \item Se \(Q_1\) e \(Q_2\) sono di pullback, allora il rettangolo
    esterno è di pullback.
  \item Se il rettangolo e \(Q_2\) sono di pullback, allora \(Q_1\) lo
    è.
  \end{enumerate}
\end{recall}

\begin{corollary}
  \label{corollary:gluing-pb-po}
  In una categoria esatta \((\cat C, \cat E)\), il rettangolo esterno in
  un diagramma
  \[
    \begin{tikzcd}
      A \ar[d, "i", swap] \ar["{f}", twoheadrightarrow, r] \ar["{PB}"
      description, phantom, dr] & B \ar[d, "j"] \ar["{PO}" description,
      phantom, dr]
      \ar["{g}", rightarrowtail, r] & C \ar["{k}", d] \\
      A' \ar["{f'}", twoheadrightarrow, r, swap] & B' \ar["{g'}",
      rightarrowtail, r, swap] & C'
    \end{tikzcd}
  \]
  è bicartesiano e
  \begin{equation}
    \begin{tikzcd}[ampersand replacement=\&, column sep=large]
      A \ar[r, "{\begin{pmatrix} i \\ gf \end{pmatrix}}"] \& A \oplus C
      \ar[r, "{\begin{pmatrix} g' f' & -k \end{pmatrix}}"] \& C'
    \end{tikzcd}
    \label{cd:gluing-pb-po-exact}
  \end{equation}
  è una successione esatta corta.
\end{corollary}

\begin{proof}
  Ne consegue dalla Proposizione~\ref{proposition:pushout-exact-bicart}
  che entrambi i quadrati sono bicartesiani. Grazie al lemma dei
  pullback (Richiamo~\ref{recall:PBLemma}), incollare i due quadrati
  produce un altro quadrato bicartesiano. Per la
  Proposizione~\ref{proposition:FromPullbackToKernel}, i morfismi
  in~\eqref{cd:gluing-pb-po-exact} formano una coppia ker-coker. Per
  concludere proviamo che questa coppia appartiene a \(\cat E\).
  \[
    \begin{pmatrix}
      g' f' & -k
    \end{pmatrix}
    =
    \begin{pmatrix}
      g' & -k
    \end{pmatrix}
    \underbrace{
      \begin{pmatrix}
        f' & 0 \\
        0 & \id_C
      \end{pmatrix}
    }_{f' \oplus \id_C}
  \]
  mostra \(\begin{pmatrix} g' f' & k \end{pmatrix}\) come una
  composizione di epimorfismi ammissibili secondo la
  Proposizione~\ref{proposition:SumExactSeqsIsExact} e la
  Proposizione~\ref{proposition:pushout-exact-bicart}.
\end{proof}

\begin{proposition}\label{proposition:PbAdmMonicIsAdmMonic}
  In una categoria esatta il pullback di un monomorfismo ammissibile
  lungo un epimorfismo ammissibile produce un monomorfismo ammissibile.
\end{proposition}

\begin{proof}
  Sia \(i : A \to B\) un monomorfismo ammissibile e \(e : B ' \to B\) un
  epimorfismo ammissibile. A causa di~\ref{item:E2op} abbiamo il
  quadrato di pullback
  % \[
  %   \begin{tikzcd}
  %     A' \ar[d, "{e'}", swap] \ar[r, "{i'}"] \ar["{PB}" description,
  %     phantom, dr] & B' \ar["{e}", twoheadrightarrow, d] \\
  %     A \ar["{i}", rightarrowtail, r, swap] & B
  %   \end{tikzcd}
  % \]
  \[
    \begin{tikzcd}
      A' \ar["{i'}", d, swap] \ar["{PB}" description, dr, phantom]
      \ar["{e'}", twoheadrightarrow, r] & A \ar["{i}", rightarrowtail, d] \\
      B' \ar["{e}", twoheadrightarrow, r, swap] & B
    \end{tikzcd}
  \]
  Quindi dobbiamo dimostrare che \(i'\) è un monomorfismo
  ammissibile.\newline Guardiamo il diagramma
  \[
    \begin{tikzcd}[row sep=tiny]
      A' \ar[dd, "{e'}", swap] \ar[r, "{i'}"] \ar["{PB}" description,
      phantom,
      ddr] & B' \ar["{e}", dd] \ar["{pe}", dr] \\
      & & C \\
      A \arrow[r, "{i}", swap] & B \ar["{p}", ur, swap]
    \end{tikzcd}
  \]
  dove \(p\) è un cokernel di \(i\). Qui \(p\) è un epimorfismo
  ammissibile e lo è anche \(pe\) per la~\ref{item:E1op}. Per vedere che
  \(i'\) è un monomorfismo ammissibile, è sufficiente dimostrare che
  \(i'\) è un kernel di \(pe\).\newline Anzitutto
  \(pei' = pie' = 0_{A'}^C\). Supponiamo che \(g':X \to B'\) sia tale
  che \(peg' = 0_X^C\) e cerchiamo un modo di costruire una freccia
  \(X \to A'\) per cui commuta il triangolo in
  \[
    \begin{tikzcd}
      A' \ar["{i'}", r] & B' \ar["{pe}", r] & C \\
      X \ar["{}", u] \ar["{g'}", ur, swap]
    \end{tikzcd}
  \]
  Essendo \(i\) kernel di \(p\), esiste una sola \(f: X \to A\) per cui
  \(eg' = if\). Applicando la proprietà universale del quadrato del
  pullback, sia \(f': X \to A'\) tale che \(e'f' = f\) e \(i'f' = g'\):
  è l'ultima identità che ci serve.\newline Vediamo l'unicità. In
  generale in una categoria qualsiasi, il pullback di un monomorfismo è
  un monomorfismo: nel nostro caso, questo significa che \(i'\) è monico
  perché \(i\) lo è. \(f'\) è l'unico morfismo tale che \(i'f' = g'\) e
  abbiamo finito.
\end{proof}

\begin{proposition}[Assioma Oscuro]
  \label{proposition:ObscureAxiom}
  In una categoria esatta, supponiamo che \(i: A \to B\) sia un morfismo
  che ammette un cokernel. Se esiste un morfismo \(j: B \to C\) in
  \(\cat C\) tale che \(ji : A \to C\) è un monomorfismo ammissibile,
  allora \(i\) è un monomorfismo ammissibile.
\end{proposition}

\begin{remark}
  Si chiama ``{\em assioma} oscuro'' perché in alcune trattazioni (in
  Quillen, per esempio) faceva parte della definizione di categoria
  esatta. Tuttavia viene provata la sua ridondanza da Yoneda e
  Keller.
\end{remark}

\begin{proof}
  Sia \(p: B \to D\) un cokernel di \(i\) e consideriamo un quadrato di
  pushout (che esiste a causa di~\ref{item:E2})
  \[
    \begin{tikzcd}
      A \ar["{i}", d, swap] \ar["{PO}" description, phantom, dr]
      \ar["{ji}", rightarrowtail, r] & C \ar["{}", d] \\
      B \ar["{}", rightarrowtail, r] & E
    \end{tikzcd}
  \]
  Inoltre, grazie alla
  Proposizione~\ref{proposition:pushout-exact-bicart}, abbiamo che
  \(\begin{psmallmatrix} i \\ ji \end{psmallmatrix}: A \to B \oplus C\)
  è un monomorfismo ammissibile. Ora osserviamo anche che
  \(\begin{psmallmatrix} \id_B & 0_C^B \\ -j & \id_C \end{psmallmatrix}:
  B \oplus C \to B \oplus C\) è un isomorfismo (l'inversa si può
  calcolare direttamente), in particolare è un monomorfismo
  ammissibile. A causa di~\ref{item:E1} anche la freccia
  \(\begin{psmallmatrix} i \\ 0_A^C \end{psmallmatrix} =
  \begin{psmallmatrix}
    \id_B & 0_C^B \\
    - j & \id_C
  \end{psmallmatrix}
  \begin{psmallmatrix}
    i \\
    ji
  \end{psmallmatrix}\)
  è un monomorfismo ammissibile. Osserviamo inoltre che \(p \oplus \id_C : B
  \oplus C \to D \oplus C\) è un cokernel di~\(\begin{psmallmatrix} i \\
    0_A^C \end{psmallmatrix}\) (infatti \(p \oplus \id_C\) è coker di \(i \oplus 0_0^C\)), quindi è un
  epimorfismo ammissibile. Considera il seguente diagramma
  \[
    \begin{tikzcd}[row sep=small, ampersand replacement=\&]
      \& B \ar["{\begin{psmallmatrix} \id_B \\
          0_B^C \end{psmallmatrix}}", dd]
      \ar["{p}", r] \& D \ar["{\begin{psmallmatrix} \id_D \\ 0_D^C \end{psmallmatrix}}", dd] \\
      A \ar["{i}", ur] \ar["{\begin{psmallmatrix} i \\ 0_A^C \end{psmallmatrix}}", dr, swap] \\
      \& B \oplus C \ar["{p \oplus \id_C}", r, swap] \& D \oplus C
    \end{tikzcd}
  \]
  Qui il quadrato di destra è un pullback, di conseguenza \(p\) è un
  epimorfismo ammissibile e \(i\) è un kernel di \(p\)
\end{proof}



\section{Alcuni teoremi classici}

\begin{proposition}\label{prop:decompose-morphisms-of-exact-sequences}
  Sia \((\cat C,\cat E)\) una categoria esatta. Un
  morfismo di successioni esatte corte (vedi definizione di \(\cat
  C^{\to\to}\))
  \[
    \begin{tikzcd}
      A' \ar["{a}", d, swap] \ar["{f'}", rightarrowtail, r] & B' \ar["{b}", d] \ar["{g'}",
      twoheadrightarrow, r] & C' \ar["{c}", d] \\
      A \ar["{f}", rightarrowtail, r, swap] & B \ar["{g}",
      twoheadrightarrow, r, swap] & C
    \end{tikzcd}
  \]
  si fattorizza attraverso una successione esatta corta \(A \to D \to C'\)
  \begin{equation}
    \begin{tikzcd}
      A' \ar["{a}", d, swap] \ar["{BC}" description, phantom, dr]
      \ar["{f'}", rightarrowtail, r] & B' \ar["{b'}", d] \ar["{g'}",
      twoheadrightarrow, r] & C' \ar["{\id_{C'}}", d] \\
      A \ar["{\id_A}", d, swap] \ar["{m}",
      rightarrowtail, r, swap] & D \ar["{b''}", d,
      swap] \ar["{BC}" description, phantom, dr] \ar["{e}", twoheadrightarrow, r] & C'
      \ar["c", d] \\
      A \ar["{f}", rightarrowtail, r, swap] & B \ar["{g}",
      twoheadrightarrow, r, swap] & C
    \end{tikzcd}
    \label{cd:decompose-morphisms-of-exact-sequences}
  \end{equation}
  in modo tale che i due quadrati contrassegnati con {\em BC}
  sono bicartesiani. In particolare, esiste un isomorfismo canonico
  \(A \cup_{A'} B' \cong B \times_{C} C'\).
\end{proposition}

\begin{proof}
  Il pushout di \(f' : A' \mono B'\) lungo \(a\) dà i morfismi
  \(m : A \mono D\) e \(b' : B' \to D\). Attraverso la proprietà
  universale di pushout, introduciamo il morfismo \(e : D \to C'\) tale
  che \(e b' = \id_{C'} g' = g'\) e \(e m = 0\) e sia \(b'': D \to B\) il
  morfismo tale che \(b'' b' = b: B' \to B\) e \(b'' m = f \id_A = f\). La
  Proposizione~\ref{proposition:pushout-exact-bicart} ci consente di
  dire che \(e\) è un cokernel di \(m\). Guardiamo il quadrato in basso
  a destra. Abbiamo
  \[
    \begin{cases}
      ceb' = cg' \\
      cem  = 0
    \end{cases}
    \qquad
    \begin{cases}
      gb''b' = gb = cg' \\
      gb''m = gf = 0
    \end{cases}
  \]
  Per la proprietà universale di pushout del quadrato in alto a
  sinistra concludiamo che \(ce = g b''\).  Sempre a causa della
  Proposizione~\ref{proposition:pushout-exact-bicart}, possiamo
  concludere che anche questo è bicartesiano.
\end{proof}

\begin{corollary}[``Five Lemma'']
  \label{cor:five-lemma}
  In una categoria esatta considera un morfismo di successioni esatte corte
  \[
    \begin{tikzcd}
      A' \ar["{a}", d, swap] \ar["{}", rightarrowtail, r] & B' \ar["{b}", d] \ar["{}", twoheadrightarrow, r] & C' \ar["{c}", d] \\
      A \ar["{}", rightarrowtail, r] & B \ar["{}", twoheadrightarrow, r] & C
    \end{tikzcd}
  \]
  Se \(a\) e \(c\) sono isomorfismi, monomorfismi ammissibili oppure epimorfismi
  ammissibili, allora anche \(b\) è dello stesso tipo.
\end{corollary}

\begin{proof}
  Assumiamo prima che \(a\) e \(c\) siano isomorfismi. Poiché gli
  isomorfismi preservano pullback e pushout e guardando la
  fattorizzazione~\eqref{cd:decompose-morphisms-of-exact-sequences}, si
  ha che \(b\) è la composizione di due isomorfismi \(B' \to D \to B\). Se
  \(a\) e \(c\) sono entrambi monici ammissibili, sempre guardando il
  diagramma~\eqref{cd:decompose-morphisms-of-exact-sequences}, si ha che
  \(b\) è la composizione di due monomorfismi ammissibili (per la
  freccia \(D \to B\) vedi la
  Proposizione~\ref{proposition:PbAdmMonicIsAdmMonic}). Il caso degli
  epimorfismi ammissibili è duale.
\end{proof}

\begin{lemma}[Isomorfismo di Noether \(C/B \cong (C/A) / (B / A)\)]
  \label{lem:c/b=(c/a)/(b/a)}
  In una categoria esatta, sia il diagramma commutativo
  \[
    \begin{tikzcd}
      A \ar["{\id_A}", d, swap] \ar["{}", rightarrowtail, r] & B
      \ar["{}", rightarrowtail, d] \ar["{}", twoheadrightarrow, r] & X
      \ar["{}", dashed, rightarrowtail, d] \\
      A \ar["{}", rightarrowtail, r] & C \ar["{}", twoheadrightarrow, d]
      \ar["{}", twoheadrightarrow, r] & Y \ar["{}", dashed, twoheadrightarrow, d] \\
      & Z \ar["{\id_Z}", r, swap] & Z
    \end{tikzcd}
  \]
  in cui le prime due righe orizzontali e la colonna centrale sono
  esatte corte. Allora la terza colonna esiste, è esatta corta e è
  unicamente determinata dalla condizione che rende il diagramma
  commutativo. Inoltre, il quadrato in alto a destra è bicartesiano.
\end{lemma}

\begin{proof}
  Il morfismo \(X \to Y\) esiste unico poiché la prima riga è esatta e la
  composizione \(A \to C \to Y\) è zero, mentre il morfismo \(Y \to Z\) esiste
  unico poiché la seconda riga è esatta e la composizione
  \(B \to C \to Z\) si annulla. Per la
  Proposizione~\ref{proposition:pushout-exact-bicart}, il quadrato
  contenente \(X \to Y\) è bicartesiano. Ne consegue che \(X \to Y\) è un
  monico ammissibile e che \(Y \to Z\) è il suo cokernel, sempre a causa
  della stessa proposizione.
\end{proof}

\begin{corollary}[Lemma \(3 \times 3\)]
  \label{cor:3x3-lemma}
  In una categoria esatta, considera un diagramma commutativo
  \[
    \begin{tikzcd}[cramped]
      A' \ar["{a}", rightarrowtail, d, swap] \ar["{f'}", r] & B'
      \ar["{b}", rightarrowtail, d]
      \ar["{g'}", r] & C' \ar["{c}", rightarrowtail, d] \\
      A \ar["{a'}", twoheadrightarrow, d, swap] \ar["{f}", r] & B
      \ar["{b'}", twoheadrightarrow, d]
      \ar["{g}", r] & C \ar["{c'}", twoheadrightarrow, d] \\
      A'' \ar["{f''}", r, swap] & B'' \ar["{g''}", r, swap] & C''
    \end{tikzcd}
  \]
  in cui le colonne sono esatte corte e si assume inoltre che una delle
  seguenti condizioni sia vera:
  \begin{enumerate}[label=(\roman*), ref=(\roman*), leftmargin=*]
  \item \label{item:3x3-lemma-1} la riga centrale e una delle righe
    esterne è esatta corta;
  \item \label{item:3x3-lemma-2} le due righe esterne sono esatte corte
    e \(gf = 0\).
  \end{enumerate}
  Allora anche la riga rimanente è esatta corta.
\end{corollary}

\begin{proof}
  Assumiamo~\ref{item:3x3-lemma-1}. Le due possibilità sono duali l'una
  all'altra, quindi dobbiamo considerare solo il caso in cui le prime
  due righe sono esatte. Applichiamo la
  Proposizione~\ref{prop:decompose-morphisms-of-exact-sequences} alle
  prime due righe per ottenere il diagramma commutativo
  \[
    \begin{tikzcd}
      A' \ar["{a}", d, swap] \ar["{BC}" description, phantom, dr]
      \ar["{f'}", rightarrowtail, r] & B' \ar["{i}", d] \ar["{g'}",
      twoheadrightarrow, r] & C' \ar["{\id_{C'}}", d] \\
      A \ar["{\id_A}", d, swap] \ar["{\bar f}", rightarrowtail, r, swap]
      & D \ar["{j}", d, swap] \ar["{BC}" description, phantom, dr]
      \ar["{\bar g}", twoheadrightarrow, r] & C'
      \ar["c", d] \\
      A \ar["{f}", rightarrowtail, r, swap] & B \ar["{g}",
      twoheadrightarrow, r, swap] & C
    \end{tikzcd}
  \]
  dove \(ji = b\). Notiamo che \(i\) e \(j\) sono monici ammissibili
  rispettivamente perché pushout di un monomorfismo ammissibile e a
  causa della Proposizione~\ref{proposition:PbAdmMonicIsAdmMonic},
  rispettivamente. Per la
  Proposizione~\ref{proposition:pushout-exact-bicart}, il morfismo
  \(i': D \to A''\) determinato da \(i'i = 0\) e \(i'\bar{f} = a'\) è un
  cokernel di \(i\) e il morfismo \(j': B \epi C''\) dato da
  \(j' = c'g = g'' b'\) è un cokernel di \(j\).\newline Se sappiamo che il
  diagramma
  \[
    \begin{tikzcd}
      B' \ar["{\id_{B'}}", d, swap] \ar["{i}", rightarrowtail, r] & D
      \ar["{j}", rightarrowtail, d] \ar["{i'}",
      twoheadrightarrow, r] & A'' \ar["{f''}", d] \\
      B' \ar["{b}", rightarrowtail, r] & B \ar["{j'}",
      twoheadrightarrow, d, swap] \ar["{b'}",
      twoheadrightarrow, r] & B'' \ar["{g''}", d] \\
      & C'' \ar["{\id_{C''}}", r, swap] & C ''
    \end{tikzcd}
  \]
  è commutativo, possiamo concludere grazie al
  Lemma~\ref{lem:c/b=(c/a)/(b/a)} che \((f'',g'')\) è una successione
  esatta corta.\newline Rimane quindi da dimostrare che
  \(f'' i' = b' j\) poiché le altre relazioni di commutatività
  \(b = ji\) e \(g''b' = j'\) sono valide per costruzione. Osserviamo il
  quadrato \(A'B'AD\). Abbiamo
  \[
    (f'' i') i = 0 = b'b = (b'j) i \quad \text{e} \quad (b'j)\bar{f} = b'f = f''
    a' = (f''i') \bar{f} .
  \]
  % che insieme a
  % \[
  %   (f''i'\bar{f})a = (f''i'i)f' = 0 \qquad \text{e} \qquad (b'j\bar{f})a =
  %   f''a'a = 0 = b'bf' = (b'ji)f'
  % \]
  Per la proprietà universale di pushout, possiamo concludere che
  \(f''i' = b'j\).
  % Per dimostrare~(ii) iniziamo formando il pushout sotto \(g'\) e
  % \(b\) in modo da avere il seguente diagramma commutativo con righe e
  % colonne esatte
  % \[
  %   \xymatrix{ A' \ar@{ >->}[r]^{f'} \ar@{=}[d] & B' \ar@{->>}[r]^{g'}
  %   \ar@{ >->}[d]^{b} \ar@{}[dr]|{\text{PO}} &
  %   C' \ar@{ >->}[d]^{k} \\
  %   A \ar@{ >->}[r]^{i} & B \ar@{->>}[r]^{j} \ar@{->>}[d]^{b'} &
  %   D \ar@{->>}[d]^{k'} \\
  %   & B'' \ar@{=}[r] & B'' }
  % \]
  % in cui il cokernel \(k'\) di \(k\) è determinato da \(k'j = b'\) e
  % \(k'k = 0\), mentre \(i = bf'\) è un kernel dell'epico ammissibile
  % \(j\), vedi la Osservazione~{\color{red} !!! FIX REF !!!} e la
  % Proposizione~{\color{red} !!! FIX REF !!!}.  La proprietà del
  % pushout del quadrato \(B'C'BD\) applicata al quadrato \(B'C'BC\)
  % produce un morfismo unico \(d': D \to C\) tale che \(d'k = c\) e
  % \(d'j = g\).  Il diagramma
  % \[
  %   \xymatrix{ C' \ar@{=}[d] \ar@{ >->}[r]^{k} & D \ar[d]^{d'}
  %   \ar@{->>}[r]^{k'} &
  %   B'' \ar@{->>}[d]^{g''} \\
  %   C' \ar@{ >->}[r]^{c} & C \ar@{->>}[r]^{c'} & C'' }
  % \]
  % ha righe esatte ed è commutativo: Infatti, \(c = d'k\) è valido per
  % costruzione di~\(d'\), mentre \(c'd' = g''k'\) segue da
  % \(c'd'j = c'g = g''b' = g''k'j\) e dal fatto che \(j\) è
  % epico. Concludiamo dalla Proposizione~{\color{red} !!! FIX REF !!!}
  % che \(DCB''C''\) è un pullback, quindi \(d'\) è un epico ammissibile
  % e anche \(g = d'j\). Il morfismo unico \(d: A'' \to D\) tale che
  % \(k'd = f''\) e \(d'd = 0\) è un kernel di \(d'\). Per la proprietà
  % del pullback di \(DCB''C''\), il diagramma
  % \[
  %   \xymatrix{ A' \ar@{=}[r] \ar@{ >->}[d]^{a} &
  %   A' \ar@{ >->}[d]^{i} \\
  %   A \ar[r]^{f} \ar@{->>}[d]^{a'} & B \ar@{->>}[r]^{g}
  %   \ar@{->>}[d]^{j} &
  %   C \ar@{=}[d] \\
  %   A'' \ar@{ >->}[r]^{d} & D \ar@{->>}[r]^{d'} & C }
  % \]
  % è commutativo poiché \(k'(da') = f''a' = b'f = k'(jf)\) e
  % \(d'(da') = 0 = gf = d'(jf)\). Nota che l'ipotesi che \(gf = 0\)
  % entra in questo punto dell'argomento.  Ne consegue dal duale della
  % Proposizione~{\color{red} !!! FIX REF !!!} che \(ABA''D\) è
  % bicartesiano, quindi \(f\) è un kernel di \(g\) per la
  % Proposizione~{\color{red} !!! FIX REF !!!}.
\end{proof}



\section{Categorie quasi-abeliane}

\begin{definition}
  Una {\em categoria quasi-abeliana} è una categoria additiva in cui:
  \begin{enumerate}[leftmargin=*]
  \item Ogni morfismo ha kernel e cokernel.
  \item Il pushout di un kernel lungo un qualsiasi morfismo è esso
    stesso un kernel e il pushout di un cokernel lungo un qualsiasi
    morfismo è esso stesso un cokernel.
  \end{enumerate}
\end{definition}

Nelle categorie abeliane sappiamo, grazie alla
Proposizione~\ref{proposition:PushotMonoIsMono}, che il pushout di un
kernel è un kernel e il pullback di un cokernel è un cokernel. E questo
dimostra che le categorie abeliane sono quasi-abeliane.

\begin{remark}
  In \(\Ban\)
  \begin{enumerate}
  \item i monomorfismi sono esattamente le applicazioni lineare limitate
    iniettive;
  \item gli epimorfismi sono esattamente le applicazioni lineari
    limitate con immagine densa;
  \item un monomorfimo è un kernel se e solo se ha immagine chiusa;
  \item un epimorfismo è un cokernel se e solo se è suriettivo.
  \end{enumerate}
\end{remark}

\begin{example}
  La categoria additiva \(\Ban\) è quasi-abeliana. Una dimostrazione di
  questo fatto si può trovare in~\cite{buehler:boundedcohomology}
  (Teorema 2.3.3) oppure in~\cite{schneiders:quasi-abelian}
  (Proposizione 3.2.4). Sia
  \[
    \begin{tikzcd}
      A \ar["{f}", d, swap] \ar["{m}", rightarrowtail, r] & B \ar["{f'}", d] \\
      A' \ar["{m'}", r, swap] & B'
    \end{tikzcd}
  \]
  un quadrato di pushout in \(\Ban\) e proviamo che \(m'\) è un
  kernel. Grazie alla
  Proposizione~\ref{proposition:FromPullbackToKernel}, il quadrato di
  pushout diventa
  \[
    \begin{tikzcd}[ampersand replacement=\&]
      A \ar["{\begin{psmallmatrix} f \\ m \end{psmallmatrix}}", r] \& A'
      \oplus B \ar["{\begin{psmallmatrix} m' & -f' \end{psmallmatrix}}", r]
      \& B'
    \end{tikzcd}
  \]
  dove la seconda freccia è cokernel della prima. A patto di dimenticare
  la struttura extra di \(\Ban\) rispetto a \(\Vect_k\), si può
  dimostrare che (vedi Osservazione precedente)
  \(\begin{psmallmatrix} m' & -f' \end{psmallmatrix}\) è cokernel di
  \(\begin{psmallmatrix} f \\ m \end{psmallmatrix}\) nella categoria
  degli spazi vettoriali. Di nuovo per la
  Proposizione~\ref{proposition:FromPullbackToKernel}, abbiamo che il
  quadrato di pushout in \(\Ban\) rimane tale in
  \(\Vect_k\). Richiamiamo che \(\Vect_k\) è una categoria abeliana:
  grazie alla Proposizione~\ref{proposition:PushotMonoIsMono}, possiamo
  concludere che \(m'\) è un monomorfismo in \(\Vect_k\) e quindi, da
  definizione di categoria abeliana, un kernel. In realtà questo vale
  anche in \(\Ban\), visto che il funtore dimenticante
  \(\Ban \to \Vect_k\) è fedele.
\end{example}

Possiamo mostrare anche che le categorie quasi-abeliane sono categorie
esatte. Più precisamente:

\begin{proposition}
  La classe \(\cat E_{\max}\) delle coppie ker-coker di una categoria
  quasi-abeliana \(\cat C\) è una struttura esatta per la categoria
  stessa.
\end{proposition}

\begin{proof}
  È chiaro che $\cat E_{\max}$ è chiusa per isomorfismi (vedi
  definizione di \(\cat C^{\to\to}\)) e che tra queste coppie ker-coker
  figurano anche coppie in cui appaiono le identità. Gli assiomi di
  pullback e pushout sono parte della definizione di categorie
  quasi-abeliane. Dimostriamo che la classe dei cokernel è chiusa sotto
  composizione. Per i kernel si ragiona dualmente.\newline Siano
  $f: A \to B$ e $g: B \to C$ due cokernel e $h := gf$. Indichiamo con
  \(i_f : F \to A\), \(i_g : G \to B\) e \(i_h : H \to A\) i kernel
  rispettivamente di \(f\), \(g\) e \(h\). Nel diagramma
  \[
    \begin{tikzcd}
      F \ar["{i_f}", dr, swap] \ar["{u}", r, dashed] & H \ar["{i_h}", d]
      \ar["{v}", r, dashed] &
      G \ar["{i_g}", d] \\
      & A \ar["{h}", dr, swap] \ar["{f}", r, swap] & B \ar["{g}", d] \\
      & & C
    \end{tikzcd}
  \]
  esistono unici due morfismi $u$ e $v$ che rendono il diagramma
  commutativo, usando la proprietà universale di kernel. Si può
  facilmente verificare che il quadrato in alto a destra è un
  pullback. Ne segue che \(v\) è un cokernel perché pullback del
  cokernel \(f\) e $u$ è il suo kernel.\newline Per concludere, mostriamo che
  \(h\) è cokernel di \(i_h\). Sia \(h' : A \to X\) tale che
  \(h'i_h = 0_H^X\). Poiché \(i_f = i_h u\), allora \(h'i_f =
  0_F^X\). Poiché \(f\) è cokernel di \(i_f\), sia \(f' : B \to X\) tale
  che \(f'f = h'\). Per costruire un morfismo \(C \to X\), mostriamo che
  \(f' i_g = 0_G^X\). Questa uguaglianza segue da
  \[
    f' i_g v = f' f i_h = h' i_h = 0_H^X
  \]
  e dal fatto che \(v\) è un epimorfismo. Per la proprietà universale di
  cokernel, sia \(g' : C \to X\) tale che \(g' g = f'\). Componendo per
  \(f\) entrambi i membri, arriviamo a \(h' = f'f = g'gf =
  g'h\). Infine, se \(h' = g''h\), ne segue che \(g'' = g'\), essendo
  \(h\) un epimorfismo.
\end{proof}






%%% Local Variables:
%%% mode: LaTeX
%%% TeX-master: "../main"
%%% TeX-engine: luatex
%%% ispell-local-dictionary: "italian"
%%% End:
