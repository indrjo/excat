
\documentclass{beamer}

\mode<presentation>

%\usepackage[T1]{fontenc}
%\usepackage[utf8]{inputenc}
%\usepackage[italian]{babel}

\usetheme{CambridgeUS}

\usepackage[no-math]{fontspec}
\usepackage{polyglossia}
\setmainlanguage{italian}

\defaultfontfeatures[HaranoAjiMincho]{
    Renderer=HarfBuzz,
    Script=Kana,
    Scale=MatchUppercase,
    UprightFont=*-Light,
    BoldFont=*-Medium,
    Extension=.otf
}

\newfontfamily\kanafont{HaranoAjiMincho}
\DeclareRobustCommand\yo{\textup{\kanafont よ}}

%\usepackage{lmodern}
\usepackage{MnSymbol}
\usepackage{mathtools}
\usepackage{booktabs}


% General notations
\newcommand\nil\varnothing
\newcommand\set[1]{\left\{#1\right\}}
\newcommand\angled[1]{\left\langle#1\right\rangle}
\newcommand\abs[1]{\left\lvert#1\right\rvert}
\newcommand\Abs[1]{\left\lVert#1\right\rVert}
\renewcommand\bar\overline
\renewcommand\tilde\widetilde
\renewcommand\hat\widehat
\newcommand\tto\Rightarrow
\newcommand\lrarr\Leftrightarrow
\newcommand\epsln\varepsilon

\newcommand\hole{\phantom\square}

% Notations pertaining Category Theory
\newcommand\cat\mathcal
\newcommand\obj[1]{\left\lvert#1\right\rvert}
\newcommand\objcat[1]{\left\lvert\cat#1\right\rvert}
\newcommand\functo[1]{\xrightarrow{\ #1\ }}
\newcommand\funcot[1]{\xleftarrow{\ #1\ }}
\newcommand\functto[1]{\xRightarrow{\ #1\ }}
\newcommand\id{\mathtt 1}
\newcommand\inv[1]{#1^{-1}}
\newcommand\op[1]{#1^\text{op}}
\newcommand\opcat[1]{\op{\cat #1}}

\newcommand\im{\operatorname{im}}
\newcommand\coker{\operatorname{coker}}

% Some categories
\newcommand\Set{\mathbf{Set}}
\newcommand\Grp{\mathbf{Grp}}
\newcommand\Ring{\mathbf{Ring}}
\newcommand\Top{\mathbf{Top}}
\newcommand\Modu{\mathbf{Mod}}
\newcommand\Ab{\mathbf{Ab}}
\newcommand\Vect{\mathbf{Vect}}
\newcommand\FDVect{\mathbf{FDVect}}
\newcommand\Ban{\mathbf{Ban}}

\newcommand\adjoint[4]{
  \begin{tikzcd}[ampersand replacement=\&]
    #1 \ar["{#2}", r, shift left] \& #3 \ar["{#4}", l, shift left]
  \end{tikzcd}
}

% This is a *temporary* command: I need it to make notes around
% the text. These notes shall be removed in the final document.
\newcommand\NotaInterna[1]{{\color{red}[#1]}}



%%%%
\newcommand\mono\rightarrowtail
\newcommand\epi\twoheadrightarrow

%%% Local Variables:
%%% mode: LaTeX
%%% TeX-master: "CT"
%%% TeX-engine: luatex
%%% End:



\title{Categorie esatte}
\author{Indrjo Dedej}
\institute[UNIPV]{Università degli Studi di Pavia}
\date{Pavia, 28/10/2025}

\pgfdeclareimage[height=1cm]{logo}{logo.png}
\logo{\pgfuseimage{logo}}

\usepackage{tikz}
\usetikzlibrary{calc, babel}
\usepackage{tikz-cd}

\begin{document}

\maketitle

\section{Preludio}

\begin{frame}[fragile]
  \frametitle{Due categorie...}
  \begin{tabular}{r|cc}
    & \(\Modu_R\) & \(\Ban\) \\
    \midrule
    oggetti & moduli su \(R\) & spazi di Banach \\
    morfismi & applicazioni lineari & applicazioni lineari limitate
  \end{tabular}

  ~\vfill
  Richiamo: un'applicazione lineare tra spazi normati
  \(f : (X, \left\lVert \cdot \right\rVert_X) \to (Y, \left\lVert \cdot
  \right\rVert_Y)\) è detta {\em limitata} se esiste \(M > 0\) per cui
  \(\left\lVert f x \right\rVert \le M \left\lVert x \right\rVert_X\) per
  ogni \(x \in X\).
\end{frame}

\begin{frame}[fragile]
  \frametitle{Kernel e cokernel}

\begin{block}{Definizione}
  Sia \(\cat C\) una categoria con {\em oggetto zero} che indichiamo con
  \(0\).
  \begin{itemize}
  \item {\em Morfismo nullo}
    \(0_A^B : \begin{tikzcd}[column sep=small] A \ar["{\exists!}", r] & 0
      \ar["{\exists!}", r] & B \end{tikzcd}\)
  \item {\em Kernel} di \(f : A \to B\): equalizzatore di
    \(\begin{tikzcd}[column sep=small] A \ar["f", shift left, r]
      \ar["{0_A^B}", shift right, swap, r] & B \end{tikzcd}\)
  
  \item {\em Cokernel} di \(f : A \to B\): coequalizzatore di
    \(\begin{tikzcd}[column sep=small] A \ar["f", shift left, r]
      \ar["{0_A^B}", shift right, swap, r] & B \end{tikzcd}\)
  \end{itemize}
  
\end{block}
\end{frame}

\begin{frame}
  \frametitle{Esempi}
  In \(\Modu_R\):
  \begin{itemize}
  \item kernel di \(f : M \to N\): inclusione \(\ker f \hookrightarrow M\)
  \item cokernel di \(f : M \to N\): proiezione canonica
    \(N \to N{/}{\im f}\)
  \end{itemize}

  In \(\Ban\):
  \begin{itemize}
  \item kernel di \(f : M \to N\): inclusione \(\ker f \hookrightarrow M\)
  \item cokernel di \(f : M \to N\): proiezione canonica
    \(N \to N{/}{\bar{\im f}}\)
  \end{itemize}

  Richiamo: \((X, \left\lVert \cdot \right\rVert)\) spazio di Banach e
  \(M \subseteq X\) chiuso \(\implies\) \((X/M, \lsem \cdot \rsem)\) spazio di
  Banach con
  \(\lsem x+M \rsem := \inf_{m \in M} \left\lVert x+m \right\rVert\)
\end{frame}

\begin{frame}
  \frametitle{Categorie preadditive}

\begin{block}{Definizione}
  Una {\em categoria preadditiva} è una categoria \(\cat C\) in cui:
  \begin{enumerate}
  \item Per ogni \(A, B \in \obj{\cat C}\) la classe \(\cat C(A, B)\) è
    dotata di un'operazione interna
    \[
      +_{A,B} : \cat C (A, B) \times \cat C (A, B) \to \cat C (A, B)
    \]
    e ha un elemento \(0_A^B : A \to B\) che lo rendono un gruppo
    abeliano.
  \item Per ogni \(A, B, C, D \in \obj{\cat C}\) e \(f : A \to B\) di
    \(\cat C\), le funzioni
    \begin{align*}
      & f_\ast := \cat C(C, f) : \cat C(C, A) \to \cat C(C, B) \\
      & f^\ast := \cat C(f, D) : \cat C(B, D) \to \cat C(A, D)
    \end{align*}
    sono omomorfismi di gruppi abeliani.
  \end{enumerate}
\end{block}
  
\end{frame}

\begin{frame}[fragile]
  \frametitle{Proprietà ed esempi}
  \begin{itemize}
  \item \(\Modu_R\) e \(\Ban\) sono categorie preadditive.
  \item Un adagio: in una categoria preadditiva con oggetto zero \(0\),
    \(f : A \to B\) è monomorfismo se e solo se
    \(\begin{tikzcd}[column sep=small, cramped] 0 \ar["{\exists!}", r] &
      A \end{tikzcd}\) è kernel di \(f\).
  \item Enunciato duale.
  \item Esempio: caratterizzazione morfismi di \(\Ban\)
  \end{itemize}

\end{frame}

\begin{frame}[fragile]
  \frametitle{Biprodotti}
  
  \begin{block}{Definizione}
    Un {\em biprodotto} in una categoria preadditiva consta di oggetti e
    frecce
    \[
      \begin{tikzcd}
        A \ar["{i_A}", r, shift right, swap] & C \ar["{p_A}", l, shift
        right, swap] \ar["{p_B}", r, shift left] & B \ar["{i_B}", l,
        shift left]
      \end{tikzcd}
    \]
    tali che
    \begin{align}
      & p_A i_A = \id_A \label{eqn:Biprod1} \\
      & p_B i_B = \id_B \label{eqn:Biprod2}\\
      & i_A p_A + i_B p_B = \id_C \label{eqn:Biprod3}
    \end{align}
  \end{block}

  \begin{block}{Proposizione}
    \((p_A, p_B)\) è un prodotto, \((i_A, i_B)\) è un coprodotto.
  \end{block}
  
\end{frame}



\begin{frame}[fragile]
  \frametitle{Categorie additive}
  \begin{block}{Definizione}
    Una {\em categoria additiva} è una categoria che ha oggetto zero e
    che per ogni coppia di oggetti \(A\) e \(B\) possiede biprodotto
    \[
      \begin{tikzcd}
        A \ar["{i_A^{AB}}", r, shift right, swap] & A \oplus B
        \ar["{p_A^{AB}}", l, shift right, swap] \ar["{p_B^{AB}}", r,
        shift left] & B \ar["{i_B^{AB}}", l, shift left]
      \end{tikzcd}
    \]
  \end{block}
  
  \begin{block}{Esempio}
    \(\Modu_R\) e \(\Ban\) sono additive.
  \end{block}

  Richiamo: norma su \(X \oplus Y\) in \(\Ban\):
  \(\left\lVert (x,y) \right\rVert_{X \oplus Y} := \left\lVert x
  \right\rVert_X + \left\lVert y \right\rVert_Y\)
\end{frame}

\begin{frame}
  \frametitle{Categorie Abeliane}
  \begin{block}{Definizione}
    Una {\em categoria abeliana} è una categoria additiva in cui:
    \begin{enumerate}
    \item Ogni morfismo ha un kernel e un cokernel.
    \item Ogni monomorfismo è un kernel e ogni epimorfismo è un
      cokernel.
    \end{enumerate}
  \end{block}

  \begin{block}{Esempi}
    \begin{itemize}
    \item \(\Modu_R\) è una categoria abeliana.
    \item \(\Ban\) è una categoria abeliana?
    \end{itemize}
  \end{block}
  
\end{frame}

\begin{frame}
  \frametitle{Categorie Abeliane}

  \begin{block}{Proprietà}
    \begin{itemize}
    \item La categorie abeliane sono finitamente complete e cocomplete.
    \item Nelle categorie abeliane, mono e epi \(\implies\) iso.
    \end{itemize}
  \end{block}

  \pause
  
  \begin{block}{Esempio}
    Quindi, no, \(\Ban\) \alert{non} è abeliana.
    \[
      C_c^0[a, b] \hookrightarrow L^p[a, b]
    \]
    è mono ed epi, ma non iso.
  \end{block}
  
\end{frame}



\section{Categorie esatte}

\begin{frame}[fragile]
  \frametitle{Verso le categorie esatte}

  \begin{block}{Proposizione}
    In una categoria abeliana, i pushout preservano i kernel e i pullback
    preservano i cokernel.
    \[
      \begin{tikzcd}
        A \ar["{f}", d, swap] \ar["{PO}" description, phantom, dr] \ar["{i}", rightarrowtail, r] & B \ar["{}", d] \\
        A' \ar["{i'}", rightarrowtail, r, swap] & B'
      \end{tikzcd}
      \qquad
            \begin{tikzcd}
        B' \ar["{}", d] \ar["{PB}" description, dr, phantom] \ar["{p'}", twoheadrightarrow, r] & C' \ar["{g}", d] \\
        B \ar["{p}", twoheadrightarrow, r, swap] & C
      \end{tikzcd}
    \]
  \end{block}

  \pause
  
  \begin{block}{Ma...}
    Anche in \(\Ban\) vale!
  \end{block}
  
\end{frame}

\begin{frame}[fragile]
  \frametitle{Categorie quasi-abeliane}

  \begin{block}{Definizione}
    Una {\em categoria quasi-abeliana} è una categoria additiva in cui:
    \begin{enumerate}
    \item Ogni morfismo ha kernel e cokernel.
    \item I pushout preservano i kernel e i pullback preservano i cokernel.
    \end{enumerate}
  \end{block}

  \begin{block}{Esempio}
    \(\Ban\) è quasi-abeliana.
  \end{block}

  \begin{block}{Proposizione}
    Abeliana \(\implies\) quasi-abeliana.
  \end{block}
  
\end{frame}

\begin{frame}[fragile]
  \frametitle{Definizione categorie esatte, I}

  \begin{block}{Definizione}
    Una {\em struttura esatta} per una categoria additiva \(\cat C\) è una
  classe \(\cat E\) di coppie ker-coker di \(\cat C\) in cui:
  \begin{itemize}
  \item Chiamiamo {\em monomorfismi ammissibili} le frecce
    \(i : A \to B\) per le quali esiste \(p : B \to C\) tale che
    \(\begin{tikzcd}[column sep=small, cramped] A \ar["{i}", r] & B
      \ar["{p}", r] & C \end{tikzcd}\) appartiene ad \(\cat E\).
  \item Chiamiamo {\em epimorfismi ammissibili} le frecce
    \(p : B \to C\) per le quali esiste \(i : A \to B\) tale che
    \(\begin{tikzcd}[column sep=small, cramped] A \ar["{i}", r] & B
      \ar["{p}", r] & C \end{tikzcd}\) appartiene ad \(\cat E\).
  \end{itemize}
  tale che...
  \end{block}
  
\end{frame}

\begin{frame}[fragile]
  \frametitle{Definizione categorie esatte, II}

  \begin{block}{Definizione}
    ... tale che:
    \begin{itemize}
  \item \(\cat E\) è chiusa per isomorfismo, cioè se
    \(\begin{tikzcd}[column sep=small, cramped] A \ar["{i}", r] & B
      \ar["{p}", r] & C \end{tikzcd}\) è in \(\cat E\) ed è isomorfo a
    \(\begin{tikzcd}[column sep=small, cramped] A' \ar["{i'}", r] & B'
      \ar["{p'}", r] & C' \end{tikzcd}\) come oggetto di
    \(\cat C^{\to\to}\), allora anche quest'ultima coppia di frecce è in
    \(\cat E\).
  \item \label{item:EO} \(\id_A : A \to A\) è un monomorfismo
    ammissibile per ogni \(A \in \obj{\cat C}\).
  \item \label{item:EOop} \(\id_A : A \to A\) è un epimorfismo
    ammissibile per ogni \(A \in \obj{\cat C}\).
  \item \label{item:E1} La composizione di due monomorfismi ammissibili
    è un monomorfismo ammissibile.
  \item \label{item:E1op} La composizione di due epimorfismi ammissibili
    è un epimorfismo ammissibile.
  \end{itemize}
  ...
  \end{block}
  
\end{frame}

\begin{frame}[fragile]
  \frametitle{Definizione categorie esatte, III}

  \begin{block}{Definizione}
    ...
    \begin{itemize}
    \item \label{item:E2} Il pushout di un monomorfismo ammissibile
      lungo un qualsiasi morfismo di \(\cat C\) esiste ed è un
      monomorfismo ammissibile.
    \item \label{item:E2op} Il pullback di un epimorfismo ammissibile
      lungo un qualsiasi morfismo di \(\cat C\) esiste ed è un
      epimorfismo ammissibile.
    \end{itemize}
    \[
      \begin{tikzcd}
        A \ar["{f}", d, swap] \ar["{PO}" description, dr, phantom] \ar["{i}", rightarrowtail, r] & B \ar["{}", d] \\
        A' \ar["{i'}", rightarrowtail, r, swap] & B'
      \end{tikzcd}
      \qquad
      \begin{tikzcd}
        B' \ar["{}", d] \ar["{PB}" description, dr, phantom] \ar["{p'}", twoheadrightarrow, r] & C' \ar["{g}", d] \\
        B \ar["{p}", twoheadrightarrow, r, swap] & C
      \end{tikzcd}
    \]
    Una {\em categoria esatta} è una coppia \((\cat C,\cat E)\) come
    sopra e gli elementi di \(\cat E\) si chiamano {\em successioni
      esatte corte}.
  \end{block}
  
\end{frame}

\begin{frame}[fragile]
  \frametitle{Alcune osservazioni}

  \begin{itemize}
  \item Una categoria è esatta perché viene munita di una struttura,
    mentre per le categorie abeliane e quasi-abeliane gli assiomi sulla
    struttura sono proprietà interne.
  \end{itemize}

  \pause
  
  \begin{block}{Proposizione}
    Per una categoria quasi-abeliana \(\cat C\) ci sono diverse
    strutture esatte:
    \begin{itemize}
    \item la classe \(\cat E_{\text{max}}\) di tutte le coppie ker-coker
      di \(\cat C\)
    \item la classe \(\cat E_{\text{split}}\) delle coppie ker-coker
      \[
        \begin{tikzcd}[column sep=small]
          A \ar["{i_A}", r] & A \oplus B \ar["{p_B}", r] & B
        \end{tikzcd}
      \]
      e coppie ker-coker isomorfe.
    \end{itemize}
  \end{block}
  
\end{frame}

\begin{frame}[fragile]
  \frametitle{Successioni esatte corte, I}
  
  \begin{block}{Proposizione}
    Sia \((\cat C,\cat E)\) una categoria esatta e sia
    \[
      \begin{tikzcd}
        A \ar["{i_A}", r, shift right, swap] & A \oplus B \ar["{p_A}",
        l, shift right, swap] \ar["{p_B}", r, shift left] & B
        \ar["{i_B}", l, shift left]
      \end{tikzcd}
    \]
    un biprodotto. Allora
    \[
      \begin{tikzcd}[row sep=tiny]
        A \ar["{i_A}", r] & A \oplus B \ar["{p_B}", r] & B \\
        B \ar["{i_B}", r] & A \oplus B \ar["{p_A}", r] & A
      \end{tikzcd}
    \]
    sono successioni esatte corte, cioè coppie ker-coker in \(\cat E\).
  \end{block}
  
\end{frame}

\begin{frame}[fragile]
  \frametitle{Successioni esatte corte, II}

  \begin{block}{Proposizione}
    ``In una categoria esatta la somma di successioni esatte corte è
    esatta corta''. Vale a dire: in una categoria esatta
    \((\cat C,\cat E)\), se
    \[
      \begin{tikzcd}[row sep=tiny, column sep=small]
        A \ar["{f}", r] & B \ar["{g}", r] & C \\
        A' \ar["{f'}", r, swap] & B' \ar["{g'}", r, swap] & C'
      \end{tikzcd}
    \]
    sono successioni esatte corte, allora lo è anche
    \[
      \begin{tikzcd}[row sep=small]
        A \oplus A' \ar["{f \oplus f'}", r] & B \oplus B' \ar["{g \oplus
          g'}", r] & C \oplus C'
      \end{tikzcd}
    \]
  \end{block}
  
\end{frame}


\begin{frame}[fragile]
  \frametitle{Successioni esatte corte, III}

  \begin{block}{Proposizione}
    In una categoria esatta consideriamo il diagramma commutativo
    \[
      \begin{tikzcd}
        A \ar["{f}", d, swap] \ar["{i}", rightarrowtail, r] & B \ar["{f'}", d] \\
        A' \ar["{i'}", rightarrowtail, r, swap] & B'
      \end{tikzcd}
    \]
    Sono equivalenti:
    \begin{itemize}
    \item Il quadrato è di pushout.
    \item \(
      \begin{tikzcd}[ampersand replacement=\&]
        A \ar["{\begin{psmallmatrix} i \\ f \end{psmallmatrix}}", r] \&
        B \oplus A' \ar["{\begin{psmallmatrix} f' &
            -i' \end{psmallmatrix}}", r] \& B'
      \end{tikzcd}
      \) è esatta corta.
    \item ...
    % \item Il quadrato è bicartesiano (cioè di pullback e di pushout).
    % \item Il quadrato è parte del diagramma commutativo
    %   \[
    %     \begin{tikzcd}[row sep=tiny]
    %       A \ar["{f}", dd, swap] \ar["{i}", rightarrowtail, r] & B
    %       \ar["{f'}", dd, swap]
    %       \ar["{p}", twoheadrightarrow, dr] \\
    %       & & C \\
    %       A' \ar["{i'}", rightarrowtail, r, swap] & B' \ar["{p'}",
    %       twoheadrightarrow, ur, swap]
    %     \end{tikzcd}
    %   \]
    %   dove \((i, p)\) e \((i', p')\) sono successioni esatte corte.
    \end{itemize}
  \end{block}
  
\end{frame}

\begin{frame}[fragile]
  \frametitle{Successioni esatte corte, III}

  \begin{block}{Proposizione}
    % In una categoria esatta consideriamo il diagramma commutativo
    % \[
    %   \begin{tikzcd}
    %     A \ar["{f}", d, swap] \ar["{i}", rightarrowtail, r] & B \ar["{f'}", d] \\
    %     A' \ar["{i'}", rightarrowtail, r, swap] & B'
    %   \end{tikzcd}
    % \]
    % Sono equivalenti:
    \begin{itemize}
    % \item Il quadrato è di pushout.
    % \item \(
    %   \begin{tikzcd}[ampersand replacement=\&]
    %     A \ar["{\begin{psmallmatrix} i \\ f \end{psmallmatrix}}", r] \&
    %     B \oplus A' \ar["{\begin{psmallmatrix} f' &
    %         -i' \end{psmallmatrix}}", r] \& B'
    %   \end{tikzcd}
    %   \) è esatta corta.
    \item ...
    \item Il quadrato è bicartesiano (cioè di pullback e di pushout).
    \item Il quadrato è parte del diagramma commutativo
      \[
        \begin{tikzcd}[row sep=tiny]
          A \ar["{f}", dd, swap] \ar["{i}", rightarrowtail, r] & B
          \ar["{f'}", dd, swap]
          \ar["{p}", twoheadrightarrow, dr] \\
          & & C \\
          A' \ar["{i'}", rightarrowtail, r, swap] & B' \ar["{p'}",
          twoheadrightarrow, ur, swap]
        \end{tikzcd}
      \]
      dove \((i, p)\) e \((i', p')\) sono successioni esatte corte.
    \end{itemize}
  \end{block}
  
\end{frame}

\begin{frame}[fragile]
  \frametitle{Ancora un risultato}

  \begin{block}{Teorema di Noether, ``\(C/B \cong (C/A) / (B / A)\)''}
    In una categoria esatta, sia il diagramma commutativo
    \[
      \begin{tikzcd}
        A \ar["{\id_A}", d, swap] \ar["{}", rightarrowtail, r] & B
        \ar["{}", rightarrowtail, d] \ar["{}", twoheadrightarrow, r] & X
        \ar["{}", dashed, rightarrowtail, d] \\
        A \ar["{}", rightarrowtail, r] & C \ar["{}", twoheadrightarrow,
        d]
        \ar["{}", twoheadrightarrow, r] & Y \ar["{}", dashed, twoheadrightarrow, d] \\
        & Z \ar["{\id_Z}", r, swap] & Z
      \end{tikzcd}
    \]
    in cui le prime due righe orizzontali e la colonna centrale sono
    esatte corte. Allora la terza colonna esiste, è esatta corta e è
    unicamente determinata dalla condizione che rende il diagramma
    commutativo. Inoltre, il quadrato in alto a destra è bicartesiano.
  \end{block}
  
\end{frame}

\begin{frame}[fragile]
  \frametitle{Un teorema di embedding, I}

  \begin{block}{Lemma di Yoneda}
    Se \(\cat C\) è una categoria localmente piccola, l'embedding di
    Yoneda \(\yo : \cat C \to [\opcat C, \Set]\) è un funtore pieno e
    fedele.
  \end{block}

  \pause
  
  \begin{block}{Domanda}
    E se \(\cat C\) è una categoria additiva?
  \end{block}
  
  Possiamo considerare questo embedding di Yoneda:
  \[
    \yo : \cat C \to [\opcat C, \Ab]
  \]

  \pause
  
  \begin{block}{Teorema di embedding}
    Sia \((\cat C, \cat E)\) una categoria esatta piccola. L'embedding
    di Yoneda \(\yo : \cat C \to [\opcat C, \Ab]\) preserva e riflette
    l'esattezza.
  \end{block}
  
\end{frame}

\begin{frame}[fragile]
  \frametitle{Un teorema di embedding, II}

  Come \(\yo : \cat C \to [\opcat C, \Set]\) incorpora \(\cat C\) nella
  categoria dei prefasci \([\opcat C, \Set]\)...

  \pause

  \bigskip
  
  ... \(\yo : \cat C \to [\opcat C, \Ab]\) incorpora una
  categoria esatta in una \alert{categoria abeliana}: la sottocategoria
  piena dei funtori esatti a sinistra.
    
\end{frame}

\end{document}

%%% Local Variables:
%%% mode: LaTeX
%%% TeX-master: t
%%% TeX-engine: luatex
%%% ispell-local-dictionary: "italian"
%%% End:
