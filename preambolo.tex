
\documentclass[ structure      = book
              , fontsize       = 11pt
              , maketitlestyle = standard
              %, partfont       = smallcaps
              %, partstyle      = parright 
              , chapstyle      = parcenter
              %, secstyle       = center
              , secfont        = roman
              %, subsecstyle    = center
              , subsecfont     = roman
              %, twocolcontents = toc
              , liststyle      = aligned
              , quotesize      = normalsize
              , headerstyle    = center
              , footnotestyle  = dotted
              ]{suftesi}

\captionsetup{ width=.8\textwidth
             , labelfont=bf
             }

% \newenvironment{tcbenum}
%   {\begin{enumerate}[leftmargin=*]}
%   {\end{enumerate}}

% \newenvironment{tcbitem}
%   {\begin{itemize}[leftmargin=*]}
%   {\end{itemize}}


\usepackage{microtype}

\usepackage[no-math]{fontspec}
\usepackage[rm,tt=false]{libertine}
\usepackage[scale=.8]{sourcecodepro}

\usepackage{polyglossia}
\setmainlanguage{italian}
\setotherlanguage[variant=uk]{english}

\usepackage[italian=quotes]{csquotes}

\usepackage{hyperref}
\hypersetup{breaklinks,hidelinks}

\usepackage{booktabs}

\usepackage[bibstyle=alphabetic,citestyle=alphabetic,pluralothers=true,autolang=langname]{biblatex}
\setcounter{biburllcpenalty}{7000}
\setcounter{biburlucpenalty}{7000}
\addbibresource{biblio.bib}

\usepackage{libertinust1math}
\usepackage{MnSymbol}
\usepackage{mathtools}
\let\underbrace\LaTeXunderbrace
\let\overbrace\LaTeXoverbrace
\usepackage[bb=stix]{mathalfa}

%\renewenvironment{pmatrix}{\begin{pmatrix*}}{\end{pmatrix*}}

\usepackage{amsthm}
\newcounter{coun} % Find a nicer name for the counter...
\counterwithin{coun}{section}
\theoremstyle{definition}
\newtheorem{definition}  [coun]{Definizione}
\newtheorem{construction}[coun]{Costruzione}
\newtheorem{lemma}       [coun]{Lemma}
\newtheorem{proposition} [coun]{Proposizione}
\newtheorem{corollary}   [coun]{Corollario}
\newtheorem{remark}      [coun]{Osservazione}
\newtheorem{notation}    [coun]{Notazioni}
\newtheorem{recall}      [coun]{Richiamo}
\newtheorem{example}     [coun]{Esempio}
\newtheorem{exercise}    [coun]{Esercizio}

\usepackage[skins, theorems, breakable, hooks]{tcolorbox}
\tcbset{breakable, boxsep=0mm}
\tcolorboxenvironment{definition}{tile,
  left=2mm, right=2mm, top=2mm, bottom=2mm, colback=gray!30!white}
\tcolorboxenvironment{lemma}{skin=emptymiddle,
  borderline west={1pt}{0pt}{gray}, top=0mm, left=1mm, right=2mm,
  bottom=0mm}
\tcolorboxenvironment{proposition}{skin=emptymiddle,
  borderline west={1pt}{0pt}{gray}, top=0mm, left=1mm, right=0mm,
  bottom=0mm}
\tcolorboxenvironment{corollary}{skin=emptymiddle,
  borderline west={1pt}{0pt}{gray}, top=0mm, left=1mm, right=0mm,
  bottom=0mm}

\counterwithin{equation}{section}

\usepackage{tikz}
\usetikzlibrary{ calc
               , babel
               , arrows.meta
               }
\tikzset{>={To[length=3pt,width=3pt]}}
\usepackage{tikz-cd}
\tikzcdset{ arrow style=math font
          , diagrams={>={To[length=3pt,width=3pt]}}
          , shorten=-2pt
          , natural/.style={shorten=2pt,Rightarrow}
          %, cramped
          }
       
          
% The first chapter is the zeroth chapter.
%\setcounter{chapter}{-1}


%%% Local Variables:
%%% mode: LaTeX
%%% TeX-master: "main"
%%% TeX-engine: luatex
%%% End:
